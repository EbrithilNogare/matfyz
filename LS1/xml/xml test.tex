\documentclass[a4paper,titlepage]{article}
\usepackage[utf8]{inputenc}   	% pro unicode UTF-8
\usepackage[czech]{babel} 		%jazyk dokumentu
\usepackage{listings}
\usepackage{color}
\usepackage[T1]{fontenc}
\usepackage{hyperref}
\usepackage{listingsutf8}
\usepackage{graphicx}
\usepackage{subcaption}
\usepackage[font=small,labelfont=bf]{caption}

\definecolor{maroon}{rgb}{0.5,0,0}
\definecolor{darkgreen}{rgb}{0,0.5,0}
\lstdefinelanguage{XML}
{
  basicstyle=\ttfamily,
  morestring=[s]{"}{"},
  morecomment=[s]{?}{?},
  morecomment=[s]{!--}{--},
  commentstyle=\color{darkgreen},
  moredelim=[s][\color{black}]{>}{<},
  moredelim=[s][\color{red}]{\ }{=},
  stringstyle=\color{blue},
  identifierstyle=\color{maroon}
}

%%%%%%%%%%%%%%%%%%%%%%%%%%%%%%%%%%%%%%%%%%%%%%%%%%%%%%%%%%%%%

\begin{document}

\section{jmenné prostory}	
	\begin{lstlisting}[language=XML]
<pricelist:offer
	xmlns:pricelist="http://www.eprice.cz/e-pricelist">
	<pricelist:item tax="22%">
		<pricelist:name>
			<bib:book
				xmlns:bib="http://www.my.org/bib">
				<bib:author>Mark Logue</bib:author>
				<bib:name>The King's Speech</bib:name>
			</bib:book>
		</pricelist:name>
		<pricelist:price curr="CZK">259</pricelist:price>
	</pricelist:item>
</pricelist:offer>
	\end{lstlisting}


\section{Xml schéma a varianty DTD. U každého DTD říct, zda bere všechny varianty XML dokumentů, které projdou XML schématem.}
	XML: varianty\\
	\begin{itemize}
		\item XHTML
		\item MathML
		\item SVG - scalable vector Graphics
		\item X3D - extensible 3D
		\item XForms
		
		\item DocBook - Creation of Technical Documentation
	
	\end{itemize}


\section{Slovně popsaný xml dokument a k němu udělat XML schéma. ( Chtěl, aby tam bylo libovolné pořadí 2 elementů) }
		
		\begin{lstlisting}[language=XML]
<?xml version="1.0" encoding="UTF-8"?>
<xs:schema xmlns:xs="http://www.w3.org/2001/XMLSchema" ...>
  <xs:element name="Inicialy" type="typ"/>	
  <xs:complexType name="typ">
    <xs:choice>
      <xs:sequence>
        <xs:element name="jmeno" type="xs:string"/>
        <xs:element name="prijmeni" type="xs:string"/>
      </xs:sequence>
      <xs:sequence>
        <xs:element name="prijmeni" type="xs:string"/>
        <xs:element name="jmeno" type="xs:string"/>
      </xs:sequence>
    </xs:choice>
  </xs:complexType>
</xs:schema>
	\end{lstlisting}
	
	
\section{ }
\section{xs:assert }
	=Invariant\\
	assert – error, when the expression does not return true\\
	Using XPath\\
	\begin{lstlisting}[language=XML]
	<xs:complexType name="Interval">
 		<xs:attribute name="min" type="xs:integer"/>
 		<xs:attribute name="max" type="xs:integer"/>
 		<xs:assert test="@min < @max"/>
	</xs:complexType>
	\end{lstlisting}
	
\section{příklad na SAX. Je to push nebo pull parser? }
	je to push (asi)
	
	Streaming \textbf{pull} parsing refers to a programming model in which a client application calls methods on an XML parsing library when it needs to interact with an XML infoset; that is, the client only gets (pulls) XML data when it explicitly asks for it.
	
	Streaming \textbf{push} parsing refers to a programming model in which an XML parser sends (pushes) XML data to the client as the parser encounters elements in an XML infoset; that is, the parser sends the data whether or not the client is ready to use it at that time.
	
\section{příklady na xpath. }
\section{Vybrat z různých xpath správné.}
\section{Vypsat dopředné a rekurzivní osy v xpath. Jaký je mezi nimi rozdíl? }
	Axis:
	\begin{itemize}
		\item self
		\item parent
		\item ancestor
		\item ancestor-or-self
		\item child
		\item descendant
		\item descendant-or-self
		\item preceding-sibling
		\item preceding
		\item following-sibling
		\item following
		\item attribute
		\item namespace
	\end{itemize}
	
	\textbf{dopředná osa} (forward axis) – obsahuje pouze kontextový uzel a uzly, které následují
za kontextovým uzlem v pořadí toku dokumentu
	\textbf{reverzní osa} (reverse axis) – obsahuje pouze kontextový uzel a uzly, které předcházejí
kontextovému uzlu v pořadí toku dokumentu\\
	\textit{zdroj: https://is.muni.cz/th/mnnx5/thesis.pdf str.14}

\section{xslt. Vytvořit tabulku pro xml dokument.  }
	\begin{lstlisting}[language=XML]
<xsl:stylesheet xmlns:xsl="http://www.w3.org/1999/XSL/Transform" version="1.0">
    <xsl:output method="html" version="4.0" encoding="UTF-8" indent="yes" doctype-public="-//W3C//DTD XHTML 1.0 Strict//EN" doctype-system="http://www.w3.org/TR/xhtml1/DTD/xhtml1-strict.dtd" />
  <xsl:template match="/">
    <html>
      <body>
        <table>
          <xsl:apply-templates select="real-estate/properties/property"/>                        
        </table>
      </body>
    </html>
  </xsl:template>
    
  <xsl:template match="properties/property">
        <tr>
            <td>
                <xsl:apply-templates select="@idProperty"/>
            </td>
        </tr>
  </xsl:template>
</xsl:stylesheet>
	\end{lstlisting}
	
	
	
\section{popsat algoritmus pro výběr šablony pro daný uzel v xslt. (Jak funguje apply templates?) }
	We can always apply only one template\\
	We take the one with the highest priority
	If it is not set, the priority is evaluated implicitly as follows:	
	\begin{itemize}
		\item 0.5: path with more than one step
		\item 0: element/attribute name
		\item -0.25: *
		\item -0.5: node(), text(), …
	\end{itemize}
	there is always a template to be applied (pre-defined, default)


\section{to samé jako 10 jen pro xquery. Vysvětlit proč to nejde pokud to nejde.  }
		\begin{lstlisting}[language=XML]
<table>
  <tr>
    <th>Id</th>
    <th>Name</th>
    <th>Features</th>
  </tr>
  {
    for $p in fn:doc('data.xml')//property
    return
      
  <tr>
    <td>{ fn:data($p/@idProperty) }</td>
    <td>{ $p/name/text() }</td>
    <td>
          { $p/features/feature[1]/text() }
          {
            for $t in $p/features/feature[position() != 1]
            return fn:concat(", ", $t/text())
          }
        </td>
  </tr>
  }

</table>
	\end{lstlisting}

\section{Dietz numbering. Ukázat jak v něm poznám vztah předek-dědic.  }
	\textbf{Preorder traversal}\\
		Child nodes of a node follow their parent node\\
	\textbf{Postorder traversal}\\
		Parent node follows its child nodes\\
	\\
	 Nechť $L(v)=(x, y) a L(u)=(x', y')$, pak:\\
        $v$ je potomek $u <=> x' < x \& y' > y$

\section{Co je xquery core? Jaký má vztah k xquery?  }
	XQuery Core defines a syntactic subset of\\
	XQuery with the same expressive power as\\
	XQuery, but without duplicities
	XQuery Core is useful mainly from the theoretical point of view.\\
		The queries are long and complex


\section{?}
\section{atributové mapování }
	Generic-tree Mapping \\
		Attribute = name of the edge\\
		\begin{quote}
		Edge \textit{attribute} (sourceID, order, type, targetID)
		\end{quote}
	
	
\section{?}
\section{něco na generické mapování  }
	\begin{itemize}
  		 \item{Edge mapping}
    		\subitem{Edge (sourceID, order, label, type, targetID)}
  		 \item{Attribute mapping}
    		\subitem{Edge \textit{attribute} (sourceID, order, type, targetID)}
  		 \item{Universal mapping}
    		\subitem{Uni (sourceID, ordera1, typea1, targetIDa1, ... orderak, typeak, targetIDak)}
  		 \item{Normalized universal mapping}
    		\subitem The universal table contains for each name just one record
 						Others (i.e. multi-value attributes) are stored in \textit{overflow tables}

	\end{itemize}
\section{SQL/XML dotaz - příklad  }
	Extension of SQL which enables to work with XML data\\
	\\
	SQL expressions -> XML values
	\begin{itemize}
  		 \item XMLELEMENT – creating XML elements
  		 \item XMLATTRIBUTES – creating XML attributes
  		 \item XMLFOREST – creating XML elements for particular tuples
  		 \item XMLCONCAT – from a list of expressions creates a single XML value
  		 \item XMLAGG – XML aggregation
	\end{itemize}
	
	Příklad:
	\begin{lstlisting}
	SELECT E.id,
 		XMLELEMENT (NAME "emp",
 			XMLATTRIBUTES (E.id AS "empid"),
 				E.first || ' ' || E.surname) AS xvalue
	FROM Employees E WHERE ...
	\end{lstlisting}
	Dlaší ukázka
	\begin{lstlisting}
	SELECT COUNT (*) FROM Children D WHERE D.parent = E.id 
	\end{lstlisting}
	
	
\section{?}


\end{document}
