\documentclass[a4paper]{article}
\usepackage[czech]{babel} %jazyk dokumentu
\usepackage[margin={1cm, 2cm}]{geometry}
\usepackage[T1]{fontenc}
\usepackage[utf8]{inputenc}   % pro unicode UTF-8
\usepackage{amsmath}
\usepackage{amssymb}
\usepackage{color}
\usepackage{graphicx}
\usepackage{hyperref}
\usepackage{listings}
\usepackage{listingsutf8}
\usepackage{mathdots}
\usepackage{multicol}
\usepackage{pgfplots}
\usepackage{tikz}
\usepackage{wrapfig}

\graphicspath{ {/} }

\def\doubleunderline#1{\underline{\underline{#1}}}

%%%%%%%%%%%%%%%%%%%%%%%%%%%%%%%%%%%%%%%%%%%%%%%%%%%%%%%%%%%%%

\begin{document}

\noindent
\textbf{Predmet: Linearni algebra 2}\\
\textbf{Ukol: 7.}\\
\textbf{Verze: 1.}\\
\textbf{Autor: David Napravnik}\\
\textbf{Prezdivka: DN}


\section*{zadani}
Spoctete Choleskeho rozklad matice A a
pouzijte ho k reseni soustavy $Ax = b$
pro vektor $b=(8,-10,30)^T$\\

\section*{reseni}
$A=
\begin{bmatrix}
	1 & -1 & 4\\
	-1 & 5 & 2\\
	4 & 2 & 26\\
\end{bmatrix}
= U^TU
$\\
$=
\begin{bmatrix}
	\cdot & 0 & 0\\
	\cdot & \cdot & 0\\
	\cdot & \cdot & \cdot\\
\end{bmatrix}
\begin{bmatrix}
	\cdot & \cdot & \cdot\\
	0 & \cdot & \cdot\\
	0 & 0 & \cdot\\
\end{bmatrix}
$\\
$
=
\begin{bmatrix}
	1 & 0 & 0\\
	-1 & \cdot & 0\\
	4 & \cdot & \cdot\\
\end{bmatrix}
\begin{bmatrix}
	1 & -1 & 4\\
	0 & \cdot & \cdot\\
	0 & 0 & \cdot\\
\end{bmatrix}
$\\
$
=
\begin{bmatrix}
	1 & 0 & 0\\
	-1 & 2 & 0\\
	4 & \cdot & \cdot\\
\end{bmatrix}
\begin{bmatrix}
	1 & -1 & 4\\
	0 & 2 & \cdot\\
	0 & 0 & \cdot\\
\end{bmatrix}
$\\
$
=
\begin{bmatrix}
	1 & 0 & 0\\
	-1 & 2 & 0\\
	4 & 3 & 1\\
\end{bmatrix}
\begin{bmatrix}
	1 & -1 & 4\\
	0 & 2 & 3\\
	0 & 0 & 1\\
\end{bmatrix}
$\\
$U^Ty=b$\\
$y=[8,-1,1]^T$\\
$Ux=y$\\
$\doubleunderline{x=[2,-2,1]^T}$\\
















\section*{zadani}
Pomoci Choleskeho rozkladu invertujte matici


\section*{reseni}
$ B =
\begin{bmatrix}
	1 & 2 & -4\\
	2 & 13 & -8\\
	-4 & -8 & 20\\
\end{bmatrix}
= U^TU
$\\
$=
\begin{bmatrix}
	1 & 0 & 0\\
	2 & 3 & 0\\
	-4 & 0 & 2\\
\end{bmatrix}
\begin{bmatrix}
	1 & 2 & -4\\
	0 & 3 & 0\\
	0 & 0 & 2\\
\end{bmatrix}
$\\
$B^{-1} = U^{-1}U^{-T}$\\
$
U^{-1}=
\begin{bmatrix}
	1 & -\frac{2}{3} & 2\\
	0 & \frac{1}{3} & 0\\
	0 & 0 & \frac{1}{2}\\
\end{bmatrix}
$\\
$
U^{-T}=
\begin{bmatrix}
	1 & 0 & 0\\
	-\frac{2}{3} & \frac{1}{3} & 0\\
	2 & 0 & \frac{1}{2}\\
\end{bmatrix}
$\\
$B^{-1} = 
\begin{bmatrix}
	\frac{49}{9} & \frac{-2}{9} & 1\\
	\frac{-2}{9} & \frac{1}{9} & 0\\
	1 & 0 & \frac{1}{4}\\
\end{bmatrix}
$\\










\section*{zadani}
Mejme $A \in \mathbb{R}^{n\times n}$ pozitivne semidefinitni
a $B\in\mathbb{R}^{n\times n}$ symetrickou.
Ukazte, ze $AB$ je diagonalizovatelna. 

\section*{reseni}
$A=LL^T$\\
$B=B^T$\\
$ABB^TA^T=LL^TBB^TL^TL$\\
$ABB^TA^T=LL^TBBL^TL$\\
$ABBA=LL^TBBL^TL$\\
$ABAB=LL^TBBL^TL$\\
$ABAB=PBBP^{-1}$\\
$(AB)^2=PBBP^{-1}$\\
$(AB)^2=PS^2P^{-1}$\\
$AB=PSP^{-1}$\\
















\section*{zadani}
Bud $V$ realny vektorovy prostor se skalnim soucinem a $w_1, ..., w_n \in V$.
Gramova matice $G\in\mathbb{R}^{n\times n}$ je definovana jako $G_{ij}=\langle w_i, w_j\rangle$.
Ukazte:\\
1. Pokud jsou vektory $w_1, ..., w_n$ linearne nezavisle, pak $G$ je pozitivne definitni\\
2. $rank(G) = dim(span(w_1, ..., w_n))$

\section*{reseni}
1.)
Dane vektory jsou linearne nezavisle prave kdyz je determinant
Gramovy matice nenulovy. Nenulovy determinant znaci,
ze matice ma $n$ nenulovych vlastnich cisel (kde $n$ je rad matice).
Coz znaci, ze matice G je pozitivne definitni.\\
\\
2.)
Pokud jsou vektory linearne nezavisle, je matice regularni.\\
Pokud jeden vektor udelame linearne zavisly na jinem,
Dimenze se nam musi snizit o jedna, nebot dimenze kernelu
o jedna stoupla prave tim, ze je jeden vektor nahraditelny.






























\end{document}