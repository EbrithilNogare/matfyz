\documentclass[a4paper]{article}
\usepackage[utf8]{inputenc}   % pro unicode UTF-8
\usepackage[czech]{babel} %jazyk dokumentu
\usepackage{listings}
\usepackage{color}
\usepackage{ mathdots }
\usepackage[T1]{fontenc}
\usepackage{amssymb}
\usepackage{hyperref}
\usepackage{listingsutf8}
\usepackage{graphicx}
\usepackage{amsmath}
\usepackage[margin={1cm, 2cm}]{geometry}

\graphicspath{ {/} }

\def\doubleunderline#1{\underline{\underline{#1}}}

%%%%%%%%%%%%%%%%%%%%%%%%%%%%%%%%%%%%%%%%%%%%%%%%%%%%%%%%%%%%%

\begin{document}

\noindent
\textbf{Predmet: Linearni algebra 2}\\
\textbf{Ukol: 4.}\\
\textbf{Verze: 1.}\\
\textbf{Autor: David Napravnik}\\
\textbf{Prezdivka: DN}

\section*{zadani}
Urcete 55. mocninu matice A

\section*{reseni}
spocteme vlastni cisla:\\
$\begin{bmatrix}
	2-\lambda & 1 & 0\\
	1 & 2-\lambda & -1\\
	1 & 1 & 1-\lambda\\
\end{bmatrix}$\\
$(2-\lambda)^2(1-\lambda)-1-1+\lambda+2-\lambda$\\
$4-4\lambda+\lambda^2-4\lambda+4\lambda^2-\lambda^3$\\
$-\lambda^3+5\lambda^2-8\lambda+4$\\
$(\lambda-2)(-\lambda^2+3\lambda-2)$\\
$-(\lambda-2)^2(\lambda-1)$\\
$\lambda_1 = 1$\\
$\lambda_2 = 2$\\
dopocteme vlastni vektory:\\
pro $\lambda_1 
\begin{bmatrix}
	2-1 & 1 & 0\\
	1 & 2-1 & -1\\
	1 & 1 & 1-1\\
\end{bmatrix}
= [1,-1,0]^T
$\\
pro $\lambda_2 
\begin{bmatrix}
	2-2 & 1 & 0\\
	1 & 2-2 & -1\\
	1 & 1 & 1-2\\
\end{bmatrix}
= [1,0,1]^T
$\\
dopocteme posledni vektor z druheho:\\
$
\begin{bmatrix}
	0 & 1 & 0\\
	1 & 0 & -1\\
	1 & 1 & -1\\
\end{bmatrix}
*v_3=[1,0,1]^T
=> v_3= [0,1,0]^T
$\\
$ R =
\begin{bmatrix}
	1 & 1 & 0\\
	-1 & 0 & 1\\
	0 & 1 & 0\\
\end{bmatrix}
$\\
$ R^{-1} =
\begin{bmatrix}
	1 & 0 & -1\\
	0 & 0 & 1\\
	1 & 1 & -1\\
\end{bmatrix}
$\\
$A^{55} = RJ^{55}R^{-1}$\\
$ J =
\begin{bmatrix}
	1 & 0 & 0\\
	0 & 2 & 1\\
	0 & 0 & 2\\
\end{bmatrix}
$\\
$ J^x =
\begin{bmatrix}
	1^x & 0 & 0\\
	0 & 2^x & 2^{x-1}x\\
	0 & 0 & 2^x\\
\end{bmatrix}
$\\
$ J^{55} =
\begin{bmatrix}
	1 & 0 & 0\\
	0 & 2^{55} & 55*2^{54}\\
	0 & 0 & 2^{55}\\
\end{bmatrix}
$\\
\\
$
\begin{bmatrix}
	1 & 1 & 0\\
	-1 & 0 & 1\\
	0 & 1 & 0\\
\end{bmatrix}
\begin{bmatrix}
	1 & 0 & 0\\
	0 & 2^{55} & 55*2^{54}\\
	0 & 0 & 2^{55}\\
\end{bmatrix}
\begin{bmatrix}
	1 & 0 & -1\\
	0 & 0 & 1\\
	1 & 1 & -1\\
\end{bmatrix}=\\
\doubleunderline{
	\begin{bmatrix}
		2^{54}*55+1 & 2^{54}*55  & -2^{54}*55+2^{55}-1\\
		2^{55}-1    & 2^{55}     & 1-2^{55} \\
		2^{54}*55   & 2^{54}*55  & 2^{55}-2^{54}*55 \\
	\end{bmatrix}
}
$









\section*{zadani}
Ukazte, ze rozklad $B = QAQ^T$, kde $A$ je diagonalni a
$Q$ ortogonalni, existuje pouze pro symetricke matice

\section*{reseni}
z definice ortogonality: $Q\in\mathbb{R}^{n\times n}; Q^T = Q^{-1}$\\
$B=QAQ^T$
\\\\
transponovanim B se transponuje i jeji rozklad\\
$B^T=(QAQ^T)^T=Q^TA^T(Q^T)^T=Q^TA^TQ=Q^TAQ$
\\\\
$Q^TA^TQ=Q^TAQ$ | nebot $A$ je diagonalni, neboli $A=A^T$ \\
$Q^TAQ = QAQ^T \Rightarrow B = B^T$ | matice $B$ musi byt nutne symetricka, pro matice, ktere nejsou symetricke to zrejme neplati $\fbox{$\phantom{5}$}$\\













\section*{zadani}
Najdete matici 3. radu, ktera ma jediny vlastni vektor $v=[1,2,3]^T$

\section*{reseni}
Abychom meli jen jeden vektor, pouzijeme libovolnou Jordanovu matici,
ktera je tvorena jen jednou Jordanovou bunkou:\\
$J=
\begin{bmatrix}
	1 & 1 & 0\\
	0 & 1 & 1\\
	0 & 0 & 1\\
\end{bmatrix}\\
$\\
pak sestavime matici $S$, aby obsahovala vektor $[1,2,3]^T$ a
byla regularni:\\
$S=
\begin{bmatrix}
	1 & 0 & 0\\
	2 & 1 & 0\\
	3 & 0 & 1\\
\end{bmatrix}\\
$\\
vyslednou matici $M$ pak ziskame pomoci rovnosti $M=SJS^{-1}$,
$S^{-1}$ existuje, nebot je $S$ regularni\\
$
\begin{bmatrix}
	1 & 0 & 0\\
	2 & 1 & 0\\
	3 & 0 & 1\\
\end{bmatrix}
\begin{bmatrix}
	1 & 1 & 0\\
	0 & 1 & 1\\
	0 & 0 & 1\\
\end{bmatrix}
\begin{bmatrix}
	1 & 0 & 0\\
	2 & 1 & 0\\
	3 & 0 & 1\\
\end{bmatrix}^{-1}
= \doubleunderline{ M =
\begin{bmatrix}
	-1 & 1 & 0\\
	-7 & 3 & 1\\
	-6 & 3 & 1\\
\end{bmatrix}
}
$












\section*{zadani}
Dokazte: Vlastni cisla antisymticka matice jsou ryze imaginarni

\section*{reseni}
necht A je antisymetricka matice a predpokladejme ze $\lambda \in \mathbb{C}$
je vlastni cislo s komplexnim vlastnim vektorem $v$\\
$Ax=\lambda x$\\
$\overline{x}^TAx=\lambda\overline{x}^Tx=\lambda||x||^2$ |  \\
$\overline{x}^TAx=(Ax)^T\overline{x}=x^TA^T\overline{x}$\\
$x^TA^T\overline{x}=-x^TA\overline{x}$\\
$A\overline{x}=\overline{\lambda}\overline{x}$\\
$-\overline{\lambda}||x^2||=\lambda||x^2||$\\
$||x||\neq0$\\
$-\overline{\lambda}=\lambda$\\
$-\overline{\lambda}=-a +ib = a+ib=\lambda$\\
coz znamena ze $\lambda$ je ryze imaginarni (nebo nula)










\section*{zadani}
Dokazte: Pokud je matice $D$ antisymetricka, pak $I+D$ je regularni (kde $I$ je jednotkova matice)

\section*{reseni}
Z predchozi ulohy vime, ze vlastni cisla antisymetricke matice jsou ryze imaginarni (nebo nulove)\\
tudiz pro kazde z nich plati: $\lambda_x + 1 \neq 0$\\
a matice radu $n$ je regularni prave kdyz ma $n$ vlastnich cisel a zadne z nich neni nulove 










\end{document}