\documentclass[a4paper]{article}
\usepackage[utf8]{inputenc}   % pro unicode UTF-8
\usepackage[czech]{babel} %jazyk dokumentu
\usepackage{listings}
\usepackage{color}
\usepackage[T1]{fontenc}
\usepackage{amssymb}
\usepackage{hyperref}
\usepackage{listingsutf8}
\usepackage{graphicx}
\usepackage{amsmath}
\usepackage[margin={1cm,2cm}]{geometry}
\usepackage{pgfplots} % render grafu


\newcommand\aug{\fboxsep=-\fboxrule\!\!\!\fbox{\strut}\!\!\!}

\newcommand{\definice}[3]{
	\setcounter{section}{#1}
	\setcounter{subsection}{#2}
	\addtocounter{subsection}{-1}
	\subsection{#3}~
}

\newcommand{\veta}[3]{
	\setcounter{section}{#1}
	\setcounter{subsection}{#2}
	\addtocounter{subsection}{-1}
	\subsection{#3}~
}

\newcommand{\dukaz}{
	\subsubsection*{dukaz}~
}

\graphicspath{ {/} }

%%%%%%%%%%%%%%%%%%%%%%%%%%%%%%%%%%%%%%%%%%%%%%%%%%%%%%%%%%%%%

\begin{document}

\textbf{skalarni soucin}\\
\begin{itemize}
    \item $||v|| \geq 0$ a $0$ nastane pouze pro $v=0$
    \item $||\alpha v|| = |\alpha| ||v||$
    \item $||u+v|| \leq ||u|| + ||v||$ 
\end{itemize}

\textbf{priklady norem:}\\
na jednotkove kruznici: (manhatonova norma)\\
$1=\sqrt{x^2+(x-y)^2+y^2}$\\
nam vykresli elipsu\\
\\
cebisevova norma nam vykresli "ctverec" kde se s rostoui odmocninou kulati rohy\\
\\
\textbf{tvrzeni:}\\
pro normy ind. skalarnich soucinem plati:\\
$ ||x-y||^2 + ||x+y||^2 = 2||x||^2 + 2||y||^2 $\\
\\
\textbf{$u$ a $v$ jsou kolme prave kdyz:}
$\langle u | v \rangle = 0$\\
\\










\end{document}