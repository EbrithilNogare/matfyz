\documentclass[a4paper]{article}
\usepackage[utf8]{inputenc}   % pro unicode UTF-8
\usepackage[czech]{babel} %jazyk dokumentu
\usepackage{listings}
\usepackage{color}
\usepackage{ mathdots }
\usepackage[T1]{fontenc}
\usepackage{amssymb}
\usepackage{hyperref}
\usepackage{listingsutf8}
\usepackage{graphicx}
\usepackage{amsmath}
\usepackage[margin={1cm, 2cm}]{geometry}

\graphicspath{ {/} }

\def\doubleunderline#1{\underline{\underline{#1}}}

%%%%%%%%%%%%%%%%%%%%%%%%%%%%%%%%%%%%%%%%%%%%%%%%%%%%%%%%%%%%%

\begin{document}

\noindent
\textbf{Predmet: Linearni algebra 2}\\
\textbf{Ukol: 1.}\\
\textbf{Verze: 1.}\\
\textbf{Autor: David Napravnik}\\
\textbf{Prezdivka: DN}

\section*{zadani}
$det[-4]$ (pochopeno jako matice $\mathbb{R}^{1\times1}$)

\section*{reseni}
$
	det \left[ \begin{matrix} 1 & 0 \\ 0 & -4 \end{matrix} \right] =
	(1*(-4)) - (0*0) = -4
$\\
$
	det \left[ \begin{matrix} 1 & 0 \\ 0 & -4 \end{matrix} \right]=
	1 * det \left[ \begin{matrix} -4 \end{matrix} \right] -
	0 * det \left[ \begin{matrix} 0 \end{matrix} \right]
$\\
$
	det \left[ \begin{matrix} -4 \end{matrix} \right] = -4
$



\section*{zadani}
$det(-2I_n)$

\section*{reseni}
jelikoz mame cisla jen na diagonale a dolni trojuhelnik je tvoren nulami, staci je vynasobit\\
$det(-2I_n) = (-2)^n$



\section*{zadani}
$
det\left[ \begin{matrix} 2 & 4 & 1 \\ 3 & 2 & 4 \\ 2 & 3 & 2 \end{matrix} \right]
$

\section*{reseni}
$
= (2*2*2) + (4*4*2) + (1*3*3) - (1*2*2) - (4*3*2) - (2*4*3)
$\\
$= -3$



\section*{zadani}
$det
\left[ \begin{matrix} 
	0 & 0 & a & b \\
	c & 0 & 0 & 0 \\
	0 & d & 0 & 0 \\
	0 & 0 & e & f
\end{matrix} \right]
$

\section*{reseni}
udelame REF (bez pravidla pro nasobeni radku konstantou)\\
$
\left[ \begin{matrix} 
	c & 0 & 0 & 0 \\
	0 & d & 0 & 0 \\
	0 & 0 & a & b \\
	0 & 0 & e & f
\end{matrix} \right]
\rightarrow
\left[ \begin{matrix} 
	c & 0 & 0 & 0 \\
	0 & d & 0 & 0 \\
	0 & 0 & a & b \\
	0 & 0 & 0 & \frac{fa-eb}{a}
\end{matrix} \right]
$\\
vynasobime prvky na diagonale a jsme hotovi\\
$= c*d*a*\frac{fa-eb}{a}=cd(fa-eb)$




\section*{zadani}
$det
\left[ \begin{matrix} 
	0 & \dots & \dots & 0 & 1 \\
	\vdots & \iddots & \iddots & 1 & 0 \\
	\vdots & \iddots & \iddots & \iddots & \vdots \\
	0 & 1 & 0 & \dots & 0 \\
	1 & 2 & 2 & \dots & 2 \\
\end{matrix} \right]
$

\section*{reseni}
stejne jako v predchozim pripade dostaneme matici do REF, \\
s tim ze budeme pouze prehazovat radky, neboli otocime ji zrcadlove.\\
$
\left[ \begin{matrix} 
	1 & 2 & 2 & \dots & 2 \\
	0 & 1 & 0 & \dots & 0 \\
	\vdots & \ddots & \ddots & \ddots & \vdots \\
	\vdots & \ddots & \ddots & 1 & 0 \\
	0 & \dots & \dots & 0 & 1 \\
\end{matrix} \right]
$\\
tim ze mame spodni trojuhlenikovou matici nulovou,\\
staci vynasobit prvky diagonaly.\\
tudiz determinant je jedna




\section*{zadani}
Bud $A\in\mathbb{R}^{m\times m}$ a $B\in\mathbb{R}^{n\times n}$.
Dokazte ze $
det
\left[ \begin{matrix} 
	A & 0 \\
	0 & B \\
\end{matrix} \right]
= det(A)det(B)
$ 

\section*{reseni}
$
\left[ \begin{matrix} 
	A & 0 \\
	0 & B \\
\end{matrix} \right] =
\left[ \begin{matrix} 
	I_n & 0 \\
	0 & B \\
\end{matrix} \right] *
\left[ \begin{matrix} 
	A & 0 \\
	0 & I_n \\
\end{matrix} \right]
$\\
$
\left[ \begin{matrix} 
	I_n & 0 \\
	0 & B \\
\end{matrix} \right] = det[B]
$\\
$
\left[ \begin{matrix} 
	A & 0 \\
	0 & I_n \\
\end{matrix} \right] = det[A]
$\\
$
det \left[ \begin{matrix} 
	A & 0 \\
	0 & B \\
\end{matrix} \right] = det[A] * det[B]
$



\section*{zadani}
Vyreste Cramerovym pravidlem nastedujici soustavu dvou rovnic v $\mathbb{Z}_5$\\
$x+y=4$\\
$2x+4y=4$


\section*{reseni}
$
A = \left[ \begin{matrix} 
	1 & 1 \\
	2 & 4 \\
\end{matrix} \right]
$\\
$
b = \left[ \begin{matrix} 
	4 \\
	4 \\
\end{matrix} \right]
$\\
$
x_i = \frac{
	det(A+(b-A_{*i})e^T_i)
}{
	det(A)
}, i = 1 , \dots , n
$\\
$det(A) = 2$\\
$
x_1 = \frac{det\left(
	\left[ \begin{matrix} 
		1 & 1 \\
		2 & 4 \\
	\end{matrix} \right] + \left(
		\left[ \begin{matrix} 
			4 \\
			4 \\
		\end{matrix} \right] -
		\left[ \begin{matrix} 
			1 \\
			2 \\
		\end{matrix} \right]
	\right)e^T_i
\right)}{
	det(A)
}
\\
x_1 = \frac{det\left(
	\left[ \begin{matrix} 
		1 & 1 \\
		2 & 4 \\
	\end{matrix} \right] + 
	\left[ \begin{matrix} 
		3 & 0 \\
		2 & 0 \\
	\end{matrix} \right]
\right)}{
	det(A)
}
$\\
$
x_1 = \frac{2}{2} = 2*2^{-1} = 1
$\\
$
x_2 = \frac{det\left(
	\left[ \begin{matrix} 
		1 & 1 \\
		2 & 4 \\
	\end{matrix} \right] + \left(
		\left[ \begin{matrix} 
			4 \\
			4 \\
		\end{matrix} \right] -
		\left[ \begin{matrix} 
			1 \\
			4 \\
		\end{matrix} \right]
	\right)e^T_i
\right)}{
	det(A)
}$\\
$
x_2 = \frac{det\left(
	\left[ \begin{matrix} 
		1 & 1 \\
		2 & 4 \\
	\end{matrix} \right] + 
	\left[ \begin{matrix} 
		0 & 3 \\
		0 & 0 \\
	\end{matrix} \right]
\right)}{
	det(A)
}
$\\
$
x_2 = \frac{1}{2} = 1*2^{-1} = 3
$\\
$x = 1, y = 3$






\section*{zadani}
Pomocí adjungované matice určete matici inverzní k matici
$
\left[ \begin{matrix} 
	a & b \\
	c & d \\
\end{matrix} \right]
$



\section*{reseni}
$A^{-1} = \frac{1}{det(A)}adj(A)$\\
$det(A) = ad-cb$\\
$adj(A) = 
\left[ \begin{matrix} 
	d & c \\
	b & a \\
\end{matrix} \right]
$\\
$
A^{-1} = 
\frac{1}{ad-cb}
\left[ \begin{matrix} 
	d & c \\
	b & a \\
\end{matrix} \right] =
\left[ \begin{matrix} 
	\frac{d}{ad-cb} & \frac{c}{ad-cb} \\
	\frac{b}{ad-cb} & \frac{a}{ad-cb} \\
\end{matrix} \right]
$









\end{document}