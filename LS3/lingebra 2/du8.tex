\documentclass[a4paper]{article}
\usepackage[czech]{babel} %jazyk dokumentu
\usepackage[margin={1cm, 2cm}]{geometry}
\usepackage[T1]{fontenc}
\usepackage[utf8]{inputenc}   % pro unicode UTF-8
\usepackage{amsmath}
\usepackage{amssymb}
\usepackage{color}
\usepackage{graphicx}
\usepackage{hyperref}
\usepackage{listings}
\usepackage{listingsutf8}
\usepackage{mathdots}
\usepackage{multicol}
\usepackage{pgfplots}
\usepackage{tikz}
\usepackage{wrapfig}

\graphicspath{ {/} }

\def\doubleunderline#1{\underline{\underline{#1}}}

%%%%%%%%%%%%%%%%%%%%%%%%%%%%%%%%%%%%%%%%%%%%%%%%%%%%%%%%%%%%%

\begin{document}

\noindent
\textbf{Predmet: Linearni algebra 2}\\
\textbf{Ukol: 8.}\\
\textbf{Verze: 1.}\\
\textbf{Autor: David Napravnik}\\
\textbf{Prezdivka: DN}


\section*{1. zadani}
Zduvodnete proc jsou nasledujici formy bilinearni a naleznete
jejich maticovou reprezentaci

\section*{reseni}
\textbf{C} nasobeni realnych cisel\\
$c:\mathbb{R}\times\mathbb{R}\rightarrow\mathbb{R}$ je dana $c(x,y) = xy$,\\
coz je nasobeni realnych cisel a\\
maticova reprezentace je podle vzorce $xy = xAy$ rovna \doubleunderline{$A=[1]$}
\\\\
\textbf{A} $a:\mathbb{R}^2\times\mathbb{R}^2\rightarrow\mathbb{R}$ danou
$a(x,y) = x_1y_1+2x_2y_1+3x_2y_2$\\
nalezneme $A$ takove, ze $xAy = x_1y_1+2x_2y_1+3x_2y_2$\\
\doubleunderline{$A = 
\begin{bmatrix}
	1 & 0 \\
	2 & 3 \\
\end{bmatrix}
$}\\
je bilinearni, nebot pro ni existuje maticova reprezentace
\\\\
\textbf{B} $b:\mathbb{Z}^n_2\times\mathbb{Z}^n_2\rightarrow\mathbb{Z}$ danou
$b(x,y) = \left(\sum^n_{i=1}ix_i \right) \cdot
\left(\sum^n_{j=1}jy_j \right)$\\
jelikoz se pohybujeme nad binarni soustavou tak sumy budou bud pricitat $x_i$ a nebo nulu.\\
vysledkem nasobeni bude operace AND, u ktereho plati:
$\sum^n_{i=1}(x_i \& y_i) = \sum^n_{i=1}(x_i) \& \sum^n_{i=1}(y_i)$ pro $x,y \in {0,1}$\\
tudiz muzeme aplikovat matici s diagonalou stridajicich se nul a jednicek\\
\doubleunderline{$A =
\begin{bmatrix}
	1 & 0 & \dots \\
	0 & 0 & \ddots \\
	\vdots & \ddots& 1 \\
	 & & & \ddots\\
\end{bmatrix}
$}







\section*{2. zadani}
Uvazte kvadratickou formu $c:\mathbb{R}^2\rightarrow\mathbb{R}$
danou predpisem $c(x)=x^2_1-6x_1x_2+9x_2^2$.
Urcete jeji maticovou reprezentaci vuci kanonicke bazi a 
bazi $B=\{(1,2)^T, (1,1)^T\}$.

\section*{reseni}
$
b(x,x) =
b_{11}x_1^2+2b_{12}x_1x_2+b_{22}x_2^2 = 
x^2_1-6x_1x_2+9x_2^2
$\\
Snadno vidime, ze 
$b_{11} = 1$, $b_{12} = b_{21} = -3$, $b_{22} = 9$\\
\doubleunderline{$A=
\begin{bmatrix}
	1&-3\\
	-3&9\\
\end{bmatrix}
$}\\
pro bazi $B$ mejme prechodovou funkci $S$ z kanonicke baze do baze $B$\\
$S = 
\begin{bmatrix}
	1&1\\
	2&1\\
\end{bmatrix}
$\\
$A_b = S^TAS$\\
\doubleunderline{$A_b =
\begin{bmatrix}
	4&2\\
	2&1\\
\end{bmatrix}
$}\\










\section*{3. zadani}
Bud $\mathbb{R}^{n\times n}$ vektorovy prostor realnych matic dimenze $n\times n$.
Definujme formu $d:\mathbb{R}^{n\times n}\times\mathbb{R}^{n\times n}\rightarrow\mathbb{R}$
predpisem $d(A,B) = trace(A^T B)$, kde $trace(C)=\sum^n_{i=1}c_{ii}$
je stopa matice. Ukazte, ze $d$ je bilinearni forma. Je $d$ symetricka?

\section*{reseni}
$d(AB)=trace(A^TB)$\\
overime podminky linearity na obou slozkach:\\
$d(\alpha u + \beta v, B) = \alpha d(u,B) + \beta d(v,B)$\\
$trace((\alpha u + \beta v)^T B) = \alpha~trace(u^TB) + \beta~trace(v^TB)$\\
\\
$d(A, \alpha u + \beta v) = \alpha d(A,u) + \beta d(A,v)$\\
$trace(A^T(\alpha u + \beta v)) = \alpha~trace(A^Tu) + \beta~trace(A^Tv)$\\
obe slozky jsou linearni, tudiz forma je bilinearni\\
\\
symetricka je, protoze je splnena podminka\\
$b(u,v) = b(v,u)~\forall v,u \in V$ jelikoz\\
$tr(A^TB) = tr(B^TA)$















\section*{4. zadani}
Necht $f$ je bilinearni forma a dale $A$ jeji maticova reprezentace vuci nejake bazi
$B$. Dokazte, nebo vyvratte,
ze vlastni cisla matice $A$ jsou nezavisle na volbe baze $B$.


\section*{reseni}
vlastni cisla matice $A$ jsou zavisle na volbe baze $B$\\
mejme baze $B$ a $B'$ a matici prechodu $S={}_B[id]_{B'}$\\
$b(u,v) =
[u]^T_BA[v]_B =
({}_B[id]_{B'} \cdot [u]_{B'})^T~A~({}_B[id]_{B'})=
[u]^T_{B'}S^TAS[v]_{B'}
$\\
tudiz se nam z matice $A$ stava matice $S^TAS$ a tyto dve matice
nebudou mit (vetsinou) stejna vlastni cisla







































\end{document}