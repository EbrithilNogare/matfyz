\documentclass[a4paper]{article}
\usepackage[utf8]{inputenc}   % pro unicode UTF-8
\usepackage[czech]{babel} %jazyk dokumentu
\usepackage{listings}
\usepackage{color}
\usepackage[T1]{fontenc}
\usepackage{amssymb}
\usepackage{hyperref}
\usepackage{listingsutf8}
\usepackage{graphicx}
\usepackage{amsmath}
\usepackage[margin={1cm,2cm}]{geometry}

\graphicspath{ {/} }

\def\doubleunderline#1{\underline{\underline{#1}}}

%%%%%%%%%%%%%%%%%%%%%%%%%%%%%%%%%%%%%%%%%%%%%%%%%%%%%%%%%%%%%

\begin{document}

\noindent
\textbf{Predmet: Mataliza 1}\\
\textbf{Ukol: 3.}\\
\textbf{Verze: 1.}\\
\textbf{Autor: David Napravnik}

\section*{zadani}
Spoctete limitu: 
$lim_{n\rightarrow \infty}\frac{2^n}{n!}$

\section*{reseni}
to ze $n!$ roste rychleji nez $2^n$ je zrejme, 
staci dokazat, ze jejich podil je limitne nulovy.\\
\\
Dokazme si nejdrive ze plati: \\
$lim_{n\rightarrow \infty}\frac{n!}{2^n} = \infty$ \\
\\
pro $n\geq6$ plati: $2^n * n < n!$
(magicke cislo 6 muzeme pouzit, protoze se budeme pohybovat v nekonecnu a $6<\infty$)\\
z toho ziskame ze $2^n < \frac{n!}{n}$ pro $n>6$\\
pak $\frac{n!}{2^n} > n$ pro $n>6$\\
tudiz 
$lim_{n\rightarrow \infty}\frac{n!}{2^n} = \infty$\\
\\
z toho dostavame ze
$\doubleunderline{lim_{n\rightarrow \infty}\frac{2^n}{n!} = 0}$


\section*{zadani}
Spoctete limitu: 
$lim_{n\rightarrow \infty} \sqrt{n}\left( \sqrt{n+1} - \sqrt{n-1} \right)$

\section*{reseni}
$$
lim_{n\rightarrow \infty} \sqrt{n \left( \sqrt{n+1} - \sqrt{n-1} \right)^2}
$$
$$
lim_{n\rightarrow \infty}
n \left( \sqrt{n+1} - \sqrt{n-1} \right)^2
$$
$$
lim_{n\rightarrow \infty}
\frac
{
	n
	\left(\sqrt{n+1} - \sqrt{n-1}\right)^2
	\left(\sqrt{n+1} + \sqrt{n-1}\right)^2
}
{\left(\sqrt{n+1} + \sqrt{n-1}\right)^2}
$$

$$
lim_{n\rightarrow \infty}
\frac
{n (A - B)^2 (A + B)^2}
{(A + B)^2};
A=\sqrt{n+1};
B=\sqrt{n-1};
$$

$$
lim_{n\rightarrow \infty}
\frac
{n (A^2 + B^2 - 2AB) (A^2 + B^2 + 2AB)}
{(A + B)^2}
$$
$$
lim_{n\rightarrow \infty}
\frac
{n \left((A^2+B^2)^2-(2AB)^2\right)}
{(A + B)^2}
$$
$$
lim_{n\rightarrow \infty}
\frac
{n (A^4+B^4-2A^2B^2)}
{A^2+B^2+2AB}
$$
$$
lim_{n\rightarrow \infty}
\frac
{n (n^2+1+2n+n^2+1-2n-2(n+1)(n-1))}
{n+1+n-1+2\sqrt{(n+1)(n-1)}}
$$
$$
lim_{n\rightarrow \infty}
\frac
{4n}
{2n+2\sqrt{n^2+1-2n}}
$$
v $\sqrt{n^2+1-2n}$ ponechame pouze nejvyssi polynom
$$
lim_{n\rightarrow \infty}
\frac
{4n}
{2n+2\sqrt{n^2}}
=lim_{n\rightarrow \infty}\frac{4n}{4n} = 1
$$
$$
\doubleunderline{
	lim_{n\rightarrow \infty} \sqrt{n}\left( \sqrt{n+1} - \sqrt{n-1} \right) = 1
}
$$









\end{document}