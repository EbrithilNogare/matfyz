\documentclass[a4paper]{article}
\usepackage[utf8]{inputenc}   % pro unicode UTF-8
\usepackage[czech]{babel} %jazyk dokumentu
\usepackage{listings}
\usepackage{color}
\usepackage[T1]{fontenc}
\usepackage{amssymb}
\usepackage{hyperref}
\usepackage{listingsutf8}
\usepackage{graphicx}
\usepackage{amsmath}
\usepackage[margin={1cm,2cm}]{geometry}

\graphicspath{ {/} }

\def\doubleunderline#1{\underline{\underline{#1}}}

%%%%%%%%%%%%%%%%%%%%%%%%%%%%%%%%%%%%%%%%%%%%%%%%%%%%%%%%%%%%%

\begin{document}

\noindent
\textbf{Predmet: Mataliza 1}\\
\textbf{Ukol: 4.}\\
\textbf{Verze: 2.}\\
\textbf{Autor: David Napravnik}\\
\textbf{Prezdivka: DN}

\section*{zadani}
Spoctete $lim_{n \rightarrow \infty} \sqrt[n]{n^2 + 1}$


\section*{reseni}
Budeme pouzivat vetu o dvou pocicajtech,
tudiz si vytvorime limity vetsi a mensi nez je zadana.\\
$
lim_{n \rightarrow \infty} \sqrt[n]{(n - 1)^2} \leq
lim_{n \rightarrow \infty} \sqrt[n]{n^2 + 1} \leq
lim_{n \rightarrow \infty} \sqrt[n]{(n + 1)^2}
$\\\\
vezmeme mensi limitu a vypocteme ji\\
$lim_{n \rightarrow \infty} \sqrt[n]{(n - 1)^2} =
lim_{n \rightarrow \infty} (\sqrt[n]{n - 1} * \sqrt[n]{n - 1})$\\
pouzijeme VOAL\\
$lim_{n \rightarrow \infty} \sqrt[n]{n - 1} * lim_{n \rightarrow \infty} \sqrt[n]{n - 1}$\\
pouzijeme lemma $lim_{n \rightarrow \infty} \sqrt[n]{n}=1$
(-1 je konstanta, kterou vzhledem k tomu, ze n jde k nekonecnu muzeme vypustit)\\
\\
to same provedeme pro vetsi limitu\\
jelikoz je to identicky postup, rovnou rekneme, ze je rovna jedne\\
\\
doplnime do vety o dvou policajtech\\
$1 \leq lim_{n \rightarrow \infty} \sqrt[n]{n^2 + 1} \leq 1$\\
\doubleunderline{$lim_{n \rightarrow \infty} \sqrt[n]{n^2 + 1} = 1$}



\section*{zadani}
Spoctete limitu posloupnosti zadane 
$a_1 = 1, a_{n+1} = \frac{a^2_n}{4} + 1$


\section*{reseni}
Nejdrive zkusme najit $a$ takove ze $a_{n+1} = a_n$\\
$a = \frac{a^2}{4} + 1$ ; $a = 2$\\
z toho vime, ze pokud posloupnost konverguje, tak to bude k dvojce.\\
\\
zjistime jak se posloupnost chova\\
$a > \frac{a^2}{4} + 1$ ; $a \in \emptyset$\\
$a = \frac{a^2}{4} + 1$ ; $a \in \{2\}$\\
$a < \frac{a^2}{4} + 1$ ; $a \in (-\infty , 2) \cup (2 , \infty)$\\
z toho vyplyva ze $\forall n \in \mathbb{N}: a_n \leq a_{n+1}$ \dots posloupnost je neklesajici\\
\\
overime ze nepreskocime dvojku az se k ni budeme priblizovat\\
$\frac{a^2_n}{4} + 1 < 2$ ; $a \in (2, \infty)$\\
coz nam rika, ze dokud bude $a_n\leq2$, tak $a_{n+1} \leq 2$\\
\\
Jelikoz $a_1 < 2$, posloupnost je neklesajici a 2 nijak 
nepreskocime, pak\\
\doubleunderline{posloupnost konverguje k 2}\\


\end{document}