\documentclass[a4paper]{article}
\usepackage[utf8]{inputenc}   % pro unicode UTF-8
\usepackage[czech]{babel} %jazyk dokumentu
\usepackage{listings}
\usepackage{color}
\usepackage[T1]{fontenc}
\usepackage{amssymb}
\usepackage{hyperref}
\usepackage{listingsutf8}
\usepackage{graphicx}
\usepackage{amsmath}
\usepackage{pgfplots}
\usepackage[margin={1cm,2cm}]{geometry}

\graphicspath{ {/} }

\def\doubleunderline#1{\underline{\underline{#1}}}

%%%%%%%%%%%%%%%%%%%%%%%%%%%%%%%%%%%%%%%%%%%%%%%%%%%%%%%%%%%%%

\begin{document}

\noindent
\textbf{Predmet: Mataliza 1}\\
\textbf{Ukol: 7.}\\
\textbf{Verze: 1.}\\
\textbf{Autor: David Napravnik}\\
\textbf{Prezdivka: DN}

\section*{Spolecne zadani}
Spoctete limity

\section*{zadani}
$\lim_{x \rightarrow 1} \frac{x^2 + 2x -1}{x^2-1}$

\section*{reseni}
Pouzijeme vetu o aritmetice limit pro nasobeni\\
$
\lim_{x \rightarrow 1} \frac{1}{x^2-1} *
\lim_{x \rightarrow 1} \frac{x^2 + 2x -1}{1}
$\\
$\lim_{x \rightarrow 1} \frac{x^2 + 2x -1}{1} = 2$\\
$\lim_{x \rightarrow 1} \frac{1}{x^2-1}$ nema reseni, proto ji vyresime zleva a zprava\\
$\lim_{x \rightarrow 1^{+}} \frac{1}{x^2-1} = \infty$\\
$\lim_{x \rightarrow 1^{-}} \frac{1}{x^2-1} = -\infty$\\
\\
$\doubleunderline{\lim_{x \rightarrow 1^+} \frac{x^2 + 2x -1}{x^2-1}} = \infty$\\
$\doubleunderline{\lim_{x \rightarrow 1^1} \frac{x^2 + 2x -1}{x^2-1}} = -\infty$








\section*{zadani}
$\lim_{x \rightarrow \pi / 4} \frac{\sqrt{\tan x}}{2 \sin^2 (x) -1}$

\section*{reseni}
Pouzijeme vetu o aritmetice limit pro nasobeni\\
$
\lim_{x \rightarrow \pi / 4} \frac{1}{2 \sin^2 (x) -1}\\
\lim_{x \rightarrow \pi / 4} \frac{\sqrt{\tan x}}{1}
$\\
$
\lim_{x \rightarrow \pi / 4} \frac{\sqrt{\tan x}}{1} = 1
\lim_{x \rightarrow \pi / 4} \frac{1}{2 \sin^2 (x) -1}$ nema reseni, proto ji vyresime zleva a zprava\\

$\lim_{x \rightarrow (\pi / 4)^+} \frac{1}{2 \sin^2 (x) - 1} = \frac{1}{2*(.5^+)-1} = \infty$\\
$\lim_{x \rightarrow (\pi / 4)^-} \frac{1}{2 \sin^2 (x) - 1} = \frac{1}{2*(.5^-)-1} = -\infty$\\
\\
$\doubleunderline{\lim_{x \rightarrow (\pi / 4)^+} \frac{\sqrt{\tan x}}{2 \sin^2 (x) -1}} = \infty$\\
$\doubleunderline{\lim_{x \rightarrow (\pi / 4)^-} \frac{\sqrt{\tan x}}{2 \sin^2 (x) -1}} = -\infty$






\section*{zadani}
$\lim_{x \rightarrow \infty} \frac{\log(1+\sqrt{x}+\sqrt[3]{x})}{\log(1+\sqrt[3]{x}+\sqrt[4]{x})}$

\section*{reseni}
Pouzijeme Lhopitalovo pravidlo, protoze $\frac{\infty}{\infty}$ je zakazana operace\\
$
\lim_{x \rightarrow \infty} \frac{f(x)}{g(x)} = 
\lim_{x \rightarrow \infty} \frac{f'(x)}{g'(x)}
$\\
vypocteme tedy derivace\\
$
\lim_{x \rightarrow \infty} \frac{
	\log(1+\sqrt{x}+\sqrt[3]{x})
}{
	\log(1+\sqrt[3]{x}+\sqrt[4]{x})
}' =
$\\
$
\lim_{x \rightarrow \infty} \frac{
	\frac{
		3x^{1/6} + 2
	}{
		6(x^{7/6} + x^{2/3} + x)
	}
}{
	\frac{
		4 x^{1/12} + 3
	}{
		12 (x^{1/3} + x^{1/4} + 1) x^{3/4}
	}
} =
$\\
$
\lim_{x \rightarrow \infty} \frac{
	(3x^{1/6}+2)12x^{3/4}(x^{1/3}+x^{1/4}+1)
}{
	(4x^{1/12}+3)6(x^{7/6}+x^{2/3}+x)	
} =
$\\
$
\lim_{x \rightarrow \infty} \frac{
	(3x^{1/6}+2)12x^{3/4}(x^{1/3}+x^{1/4}+1)
}{
	(4x^{1/12}+3)6(x^{7/6}+x^{2/3}+x)	
} =
$\\
$
\lim_{x \rightarrow \infty} \frac{
	(3x^{2/12}+2)12x^{9/12}(x^{4/12}+x^{3/12}+1)
}{
	(4x^{1/12}+3)6(x^{14/12}+x^{8/12}+x)	
} =
$\\
odebereme konstanty a male mocniny u scitani, nebot jdeme k nekonecnu\\
$
\lim_{x \rightarrow \infty} \frac{
	(3x^{2/12})  12x^{9/12}   (x^{4/12})
}{
	(4x^{1/12}) 6 (x^{14/12})	
} =
$\\
$
\lim_{x \rightarrow \infty} \frac{
	3*12*x^\frac{2+9+4}{12}
}{
	4*6*x^\frac{1+14}{12}	
} =
$\\
$
\lim_{x \rightarrow \infty} \frac{
	3x^\frac{15}{12}
}{
	2x^\frac{15}{12}	
} =
$\\
$
\doubleunderline{\frac{3}{2}}
$\\









\section*{zadani}
Urcete extremy funkce
$\sqrt[x]{x}$
definovane na kladnych realnych cislech

\section*{reseni}

pomoci derivace najdeme kde ma funkce vrchol\\
$\sqrt[x]{x}' = x^{\frac{1}{2}-2} (1-\log(x))$\\
$x^{\frac{1}{2}-2} (1-\log(x)) = 0$\\
$x = e$\\
\\
najdeme limitu u nuly a v nekonecnu\\
$\lim_{x \rightarrow 0^+} \sqrt[x]{x} = 0$\\
$\lim_{x \rightarrow \infty} \sqrt[x]{x} = 1$\\
\\
limita nabiva maxima $y=e^e$ v bode $x=e$ a\\
minima $y=0$ v bode $x=0$



\begin{tikzpicture}
	\begin{axis}[
		axis lines = left,
		xlabel = $x$,
		ylabel = {$f(x)$},
		ymajorgrids=true,
    	xmin=0, xmax=10,
		ymin=-0.5, ymax=2,
	]


	\addplot [
		domain=0:10, 
		samples=1000, 
		color=red,
	]
	{x^(1/x)};	
	\addlegendentry{$\sqrt[x]{x}$}
	

	\addplot [
		domain=0:10, 
		samples=1000, 
		color=blue,
	]
	{-x^(-2 + 1/x) * (-1 + ln(x))};	
	\addlegendentry{$(\sqrt[x]{x})'$}
	
	
	\end{axis}
	\end{tikzpicture}









\end{document}