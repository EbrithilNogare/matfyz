\documentclass[a4paper]{article}
\usepackage[utf8]{inputenc}   % pro unicode UTF-8
\usepackage[czech]{babel} %jazyk dokumentu
\usepackage{listings}
\usepackage{color}
\usepackage[T1]{fontenc}
\usepackage{amssymb}
\usepackage{hyperref}
\usepackage{listingsutf8}
\usepackage{graphicx}
\usepackage{amsmath}
\usepackage[margin={1cm,2cm}]{geometry}

\graphicspath{ {/} }

\def\doubleunderline#1{\underline{\underline{#1}}}

%%%%%%%%%%%%%%%%%%%%%%%%%%%%%%%%%%%%%%%%%%%%%%%%%%%%%%%%%%%%%

\begin{document}

\noindent
\textbf{Predmet: Mataliza 1}\\
\textbf{Ukol: 6.}\\
\textbf{Verze: 1.}\\
\textbf{Autor: David Napravnik}\\
\textbf{Prezdivka: DN}

\section*{zadani}
Spoctete limitu posloupnosti $lim_{n \rightarrow \infty}\log(1+\frac{1}{\sqrt{n}})$

\section*{reseni}
Pouzijeme vetu o \textit{limite slozene funkce}\\
$
	lim_{n \rightarrow \infty}\log(1+\frac{1}{\sqrt{n}}) =
	lim_{y \rightarrow y_0}log(y)\\
	y = lim_{n \rightarrow \infty}(1+\frac{1}{\sqrt{n}}) = 1\\
	lim_{y \rightarrow 1}\log(y)\\
	\doubleunderline{
		lim_{n \rightarrow \infty}\log(1+\frac{1}{\sqrt{n}}) = 0
	}\\
$





\end{document}