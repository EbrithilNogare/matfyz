\documentclass[a4paper]{article}
\usepackage[utf8]{inputenc}   % pro unicode UTF-8
\usepackage[czech]{babel} %jazyk dokumentu
\usepackage{listings}
\usepackage{color}
\usepackage[T1]{fontenc}
\usepackage{amssymb}
\usepackage{hyperref}
\usepackage{listingsutf8}
\usepackage{graphicx}
\usepackage{amsmath}
\usepackage[margin={1cm,2cm}]{geometry}

\graphicspath{ {/} }

%%%%%%%%%%%%%%%%%%%%%%%%%%%%%%%%%%%%%%%%%%%%%%%%%%%%%%%%%%%%%

\begin{document}

\noindent
\textbf{Predmet: Mataliza 1}\\
\textbf{Ukol: 1.}\\
\textbf{Verze: 1.}\\
\textbf{Autor: David Napravnik}

\section*{zadani}
$X \subseteq \mathbb{R}:$ nekonecna t. ze\\
$\exists a,b,c \in \mathbb{R}: c>0 ~\&~ \forall x,y \in X,
(x\neq y) => (a < x ~\&~ y<b ~\&~ |x-y|>c)$\\
hint: interval to nebude

\section*{reseni}
$X$ muze byt napr.\\
posloupnost definovana jako $a_n = \frac{1}{2^n}, n\in\mathbb{N}$\\
\\
overme podminky:\\
\begin{itemize}
	\item $\exists a \in \mathbb{R}~\forall x \in X: a<x$\\
	posloupnost X ma infimum rovno 0, tudiz plati pro $a=0$
	\item $\exists b \in \mathbb{R}~\forall y \in X: y<b$\\
	posloupnost X ma supremum rovno 1, tudiz plati pro $b=1$
	\item $\forall x,y \in X:|x-y|>0$ (BUNO vynechame $c$)\\
	jelikoz je posloupnost definovana funkci
	$f(n) = \frac{1}{2^n}$\\
	tak je prosta a tudiz neexistuji ruzne $n_1$ a
	$n_2$ takove, ze $f(n_1) = f(n_2)$\\
	tudiz bude jejich rozdil vzdy nenulovy.
\end{itemize}





\end{document}