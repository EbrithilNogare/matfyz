\documentclass[a4paper]{article}
\usepackage[czech]{babel} %jazyk dokumentu
\usepackage[margin={1cm,2cm}]{geometry}
\usepackage[T1]{fontenc}
\usepackage[utf8]{inputenc}   % pro unicode UTF-8
\usepackage{amsmath}
\usepackage{amssymb}
\usepackage{color}
\usepackage{graphicx}
\usepackage{hyperref}
\usepackage{listings}
\usepackage{listingsutf8}
\usepackage{multicol}
\usepackage{pgfplots}
\usepackage{tikz}
\usepackage{wrapfig}

\graphicspath{ {/} }

\def\doubleunderline#1{\underline{\underline{#1}}}

%%%%%%%%%%%%%%%%%%%%%%%%%%%%%%%%%%%%%%%%%%%%%%%%%%%%%%%%%%%%%

\begin{document}

\noindent
\textbf{Predmet: Mataliza 1}\\
\textbf{Ukol: 11.}\\
\textbf{Verze: 1.}\\
\textbf{Autor: David Napravnik}\\
\textbf{Prezdivka: DN}


\section*{zadani}
plocha utvaru ohraniceneho parabolou $y^2=x$ a primkou $y=x-2$

\section*{reseni}
\begin{wrapfigure}{r}{0.5\linewidth}
	\begin{tikzpicture}[scale=0.50]
		\begin{axis}[
			axis lines = left,
			xlabel = $x$,
			ylabel = {$y$},
			ymajorgrids=true,
			xmin=-2, xmax=10,
			ymin=-5, ymax=5,
		]	
	
		\addplot [
			domain=-2:10, 
			samples=100, 
			color=red,
		]
		{x-2};	
		\addlegendentry{$y=x-2$}
			
		\addplot [
			domain=0:10, 
			samples=100, 
			color=blue,
		]
		{sqrt(x)};	
		\addplot [
			domain=0:10, 
			samples=100, 
			color=blue,
		]
		{-sqrt(x)};	
		\addlegendentry{$y^2=x$}
				
		\end{axis}
	\end{tikzpicture}
	\begin{tikzpicture}[scale=0.50]
		\begin{axis}[
			axis lines = left,
			xlabel = $x$,
			ylabel = {$y$},
			ymajorgrids=true,
			ymin=-2, ymax=10,
			xmin=-5, xmax=5,
		]	
	
		\addplot [
			domain=-5:7, 
			samples=100, 
			color=red,
		]
		{x+2};	
		\addlegendentry{$y=x+2$}
			
		\addplot [
			domain=-4:4, 
			samples=100, 
			color=blue,
		]
		{x^2};	
		\addlegendentry{$y=x^2$}
				
		\end{axis}
	\end{tikzpicture}
\end{wrapfigure}
nejdrive si problem obratime prohozenim $x$ a $y$\\
funkce se protinaji v $x=-1$ a $x=2$\\
$\int x+2 ~dx= \frac{x^2}{2} + 2x + C$\\
$\int_{-1}^{2} x+2 ~dx= \frac{15}{2}$\\
$\int x^2 ~dx= \frac{x^3}{3} + C$\\
$\int_{-1}^{2} x^2 ~dx= 3$\\
\\
ohranicena plocha je velka $\frac{15}{2}-3=\doubleunderline{\frac{9}{2}}$







\section*{zadani}
\begin{wrapfigure}{r}{0.5\linewidth}
	\begin{tikzpicture}[scale=0.50]
		\begin{axis}[
			axis lines = left,
			xlabel = $x$,
			ylabel = {$y$},
			ymajorgrids=true,
			xmin=.5, xmax=3,
			ymin=-1, ymax=2,
		]		
		\addplot [
			domain=1:2.7182, 
			samples=100, 
			color=red,
		]
		{ln(x)};	
		\addlegendentry{$\ln x$}
	\end{axis}
	\end{tikzpicture}
\end{wrapfigure}
plocha utvaru ohraniceneho krivkou funkce $\ln x$,
osou $x$ a primkou $x=e$

\section*{reseni}
$\int \ln x ~dx = x(\ln x -1)+C$\\
$\int_1^e \ln x ~dx = 1$\\








\section*{zadani}
objem telesa vznikleho z utvaru b) rotaci kolem osy x

\section*{reseni}
$\int \pi(\ln(x))^2 dx = \pi (2x - 2x \log(x) + x \log^2(x))$\\ 
$\int_0^e \pi(\ln(x))^2 dx = \pi e$\\ 










\section*{zadani}
objem komoleho rotacniho kuzele s vyskou $v$ a polomery podstav
$r$ a $R$

\section*{reseni}
plocha 2D telesa:\\
$vr+\frac{(R-r)*v}{2} = \frac{v(r+R)}{2}$\\
obtocime kolem osy $y$\\
$\pi v(\frac{(r+R)}{2})^2$







































\end{document}