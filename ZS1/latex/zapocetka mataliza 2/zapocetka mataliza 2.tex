\documentclass[a4paper,titlepage]{article}
\usepackage[utf8]{inputenc}   % pro unicode UTF-8
\usepackage[czech]{babel} %jazyk dokumentu
\usepackage{listings}
\usepackage{color}
\usepackage[T1]{fontenc}
\usepackage{hyperref}
\usepackage{listingsutf8}
\usepackage{graphicx}
\usepackage{lipsum}
\usepackage{amssymb}
\usepackage{geometry}


 \geometry{
 a4paper,
 left=30mm,
 top=20mm,
 }

\graphicspath{ {/} }

\title{}
\author{David Nápravník}
\date{2018}

%%%%%%%%%%%%%%%%%%%%%%%%%%%%%%%%%%%%%%%%%%%%%%%%%%%%%%%%%%%%%

\begin{document}
\noindent E. Zápočtový test - MA I. (2.2.2018) \\
David Nápravník\\
\\
\setcounter{section}{1}





\section{Mějme následující číselnou řadu.}
$$
\sum^{\infty}_{n=1}\frac{3^n+(-1)^n*2^n*n}{4^n + (-1)^n*n}
$$
Podíváme se zda $\lim a_n = 0$
$$
\lim_{n\to \infty}\frac{3^n+(-1)^n*2^n*n}{4^n + (-1)^n*n}
$$
Pro a >> 1 $ \lim_{n\to \infty} a^n > \lim_{n\to \infty} n $\\
A jelikož $4^n >> 3^n + 2^n$ tak
$$
\lim_{n\to \infty}\frac{3^n}{4^n} = 0
$$
Srovnávacím kritériem můžeme odstranit $(-1)^n$ 
$$
\sum^{\infty}_{n=1}\frac{3^n+2^n*n}{4^n - n}  \geq 
\sum^{\infty}_{n=1}\frac{3^n-2^n*n}{4^n - n}
$$
jelikož se nedostaneme do záporných hodnot tak pokud nastane neabsolutní konvergence bude zároveň i absolutní.\\
\\
Podílové kritérium
$$
\lim_{n \to \infty}\frac{\frac{3^{n+1}+2^{n+1}*(n+1)}{4^{n+1} - (n+1)}}{\frac{3^n+2^n*n}{4^n - n}}
$$
$$
\lim_{n \to \infty}\frac{3^{n+1}+2^{n+1}*(n+1)}{\frac{(4^{n+1}-(n+1))*(3^n+2^n+n)}{4^n - n}}
$$
vydělíme a rovnou vynecháme zlomky rovné nule v limité x k nekonečnu
$$
\lim_{n \to \infty}\frac{3^{n+1}}{\frac{(4^{n+1})*(3^n)}{4^n}} =3/4 \to k = \frac{3}{4} \to k <1 \to Konverguje
$$

Podílové kritérium Konverguje a tudíž platí srovnávací kritérium a tudíž původní řada \textbf{konverguje absolutně i neabsolutně}.








\end{document}