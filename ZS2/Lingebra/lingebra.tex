\documentclass[a4paper]{article}
\usepackage[utf8]{inputenc}   % pro unicode UTF-8
\usepackage[czech]{babel} %jazyk dokumentu
\usepackage{listings}
\usepackage{color}
\usepackage[T1]{fontenc}
\usepackage{hyperref}
\usepackage{listingsutf8}
\usepackage{graphicx}
\usepackage{lipsum}

\graphicspath{ {/} }

%%%%%%%%%%%%%%%%%%%%%%%%%%%%%%%%%%%%%%%%%%%%%%%%%%%%%%%%%%%%%

\begin{document}
\section{náshhhobení matic}



\section{regulární matice}
pouze u čtvercovejch\\
2 vety, dukaz\\

$
A,B\in R^{n*m} reg \rightarrow A B reg\\
A,B\in R^{n*m} ...
$\\

singulární matice


\section{elementární matice}
1)vynásobit i-tého číslem $\alpha\neq 0$\\
$
\left(\begin{array}{ccccc}
  1 &   &  ř &   & 0 \\
     & .. &   &  .. &   \\
     &   & \alpha   &   &   \\
     & .. &   & ..  &   \\
 0  &   &   &   &  1 
\end{array}\right)
=I+(\alpha=1)*e_ie_i^T
$\\

2) přičtení $\alpha$  násobku i-tého členu k j-tému\\



3) výměna i->j\\



věta: \\
$A\in R^{m*n} \exists reg Q\in R^{n*m}$
$Q*A=RREF(A)$



\section{inverzní matice}
$A^{-1}$
věta: \\
Buď $A\in R^{n*m}$\\
jeli A reg, pak $A^{-1}$ existuje a je jednoznačně naopak, existuje-li $A^{-1}$ pak A reg
\subsection{jednoznačnost}
důkaz sporem:\\
$A^{-1} \neq B$\\
$(BA)*A^{=1} = B(A*A^{=1}) = BAA^{-1} = B$
\\
\subsection{výpočet inverzní matice}
algoritmus




\subsection{Matice a lineární zobrazení}
věta (jednoznačnost stačí):\\
$A,B\in R^{n*m}$
$AB=I \rightarrow A,B$
$A^{-1}=B, B^{-1}=A$





\end{document}