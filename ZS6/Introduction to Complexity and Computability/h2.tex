\documentclass[a4paper]{article}
\usepackage[utf8]{inputenc}
\usepackage{amsmath}
\usepackage{amsfonts}
\usepackage{amssymb}
\usepackage{hyperref}
\setlength{\parindent}{0in}
\usepackage{fancyhdr}
\usepackage[
    left   = 1.0 in,
    right  = 1.0 in,
    top    = 1.5 in,
    bottom = 1.5 in,
]{geometry}

\usepackage[czech]{babel} % recommended if you write in Czech

\begin{document}

\pagestyle{fancy}
\rhead{David Nápravník}

\section*{2. HW}
\subsection*{7 a}

Mejme kod turingova stroje M' jenz pridava k M tyto veci:
\begin{itemize}
    \item symboly \^{} (zacatek dat a pasce) a \$ (konec dat na pasce) do paskove abecedy
\end{itemize}
A nahradime
\begin{itemize}
    \item operaci L za prechodove funkce:  
    \begin{itemize}
        \item (\^{}, L, $\sum_i$) $\rightarrow$ zapis $\sum_i$ $\rightarrow$ proved L $\rightarrow$  zapis \^{} $\rightarrow$  proved R  
        \item (\_, L, $\sum_i$) $\rightarrow$ zapis $\sum_i$  
    \end{itemize} 
    \item operaci R za prechodove funkce: 
    \begin{itemize}
        \item (\$, L, $\sum_i$) $\rightarrow$ zapis $\sum_i$ $\rightarrow$ proved R $\rightarrow$  zapis \$ $\rightarrow$  proved L 
        \item (\_, L, $\sum_i$) $\rightarrow$ zapis $\sum_i$  
    \end{itemize} 
\end{itemize}
    
Zacneme simulovat vstup x na automatu M'.
\begin{itemize}
    \item Pokud se zastavi tak zkontrolujeme zda symboly \^{} a \$ jsou na pasce vedle sebe.
    \begin{itemize}
        \item Pokud jsou, tak paska po skonceni je prazdna a automat se ukocil $\downarrow$ se vstupem x.
        \item Pokud nejsou, tak paska neni prazdna a automat se ukoncil $\downarrow$.
    \end{itemize}
    \item Pokud se nezastavi tak nic nevime.
\end{itemize}

Tudiz pokud se zastavi, tak vime zda paska je po skonceni prazdna, ale
ne vzdy se automat zastavi, proto se jedna o castecne rozhodnutelny problem.


\subsection*{7 b}

Vezmeme automat M' z predchozi ulohy a vyvorime totozny automat M'' jen s rozdilem, ze M'' se zastavi po n krocich a neprijme.\\
Udelame si matici kde radky jsou n (pocet kroku do zastaveni) a
sloupce budou vsechny vstupni retezce (\textbf{shortlex}).\\
pak matici prochazime cik-cak (kazdy 2D index je prevoditelny na 1D index).\\
\begin{itemize}
    \item Pokud se M'' zastavi a prijima, tak hledany vstup x existuje. (tez plati, ze pokud x automat zastavi).\\
    \item Pokud se nezastavi, tak opet nevime nic. (proto se jedna pouze o castecne rozhodnutelny problem) 
\end{itemize}


\subsection*{8}

Mejme jazyk $L_u$ a ukazme si, ze je prevoditelny na jazyk PP.\\
Jakmile je jazyk $L_u$ prijimany v prijimajicim stavu, tak jeste do automatu pridame mazaci instrukce,
ktere nam pasku vyprazdni (smerem vlevo i vpravo) a dostaneme tak jazyk PP.\\
Avsak podle definic vime, ze $L_u$ je nerozhodnutelny a jelikoz je jazyk PP pouze podmnozina jazyka $L_u$, tak i jazyk PP musi byt nerozhodnutelny.



\end{document}