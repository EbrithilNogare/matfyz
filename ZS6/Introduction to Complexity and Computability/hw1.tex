\documentclass[a4paper]{article}
\usepackage[utf8]{inputenc}
\usepackage{amsmath}
\usepackage{amsfonts}
\usepackage{amssymb}
\setlength{\parindent}{0in}
\usepackage{fancyhdr}

\begin{document}

\pagestyle{fancy}
\rhead{David Nápravník}

\section*{1. HW}
\subsection*{9}
Mejme turinguv stroj $M$ s abecedou \{a..z\}, instrukcemi \{L, R\} a stavy \{stavA..stavZ\}.

Pak $M'$ bude turinguc stroj s abecedou \{a..z\}, instrukcemi \{L, R\} a stavy \{stavA, stavA-L, stavA-R .. stavZ, stavZ-L, stavZ-R\}.  

Neboli pronasobime stavy a instrukce, tim se pocet stavu ztrojnasobi a stav si bude pamatovat i nasledujici instrukci.  
Pokud vydime stav s instrukci, tak instrukci vykoname a stav zmenime na totozny bez instrukce.
Pokud vydime stav bez instrukce tak prepiseme znak na pasce a nastavime novy stav s instrukci, jenz bychom normalne vyzadovali.

\subsection*{10}
Prvne si vytvorime Turinguv stroj s polovicni paskou tak, ze smichame kladne a zaporne pozice.  
.. -3 -2 -1 0 1 2 3 .. $\rightarrow$ -0 0 -1 1 -2 2 -3 3 ..  

misto instrukce R v kladnnych a L v zapornych dame instrukci RR,  

misto instrukce L v kladnnych a R v zapornych dame instrukci LL.

Nuly pro nas budou obratniky, takze 
0\_R $\rightarrow$ RR, 
0\_L $\rightarrow$ R, 
-0\_R $\rightarrow$ RRR, 
-0\_L $\rightarrow$ RR, 


\end{document}