\documentclass[a4paper]{article}
\usepackage[utf8]{inputenc}
\usepackage{amsmath}
\usepackage{amsfonts}
\usepackage{amssymb}
\usepackage{hyperref}
\setlength{\parindent}{0in}
\usepackage{fancyhdr}
\usepackage[
    left   = 1.0 in,
    right  = 1.0 in,
    top    = 1.5 in,
    bottom = 1.5 in,
]{geometry}

\usepackage[czech]{babel} % recommended if you write in Czech

\begin{document}

\pagestyle{fancy}
\rhead{David Nápravník}

\section*{1. HW}
\subsection*{9}
Mejme turinguv stroj $M$ s abecedou \{a..z\}, instrukcemi \{L, R\} a stavy \{stavA..stavZ\}.

Pak $M'$ bude turinguv stroj s abecedou \{a .. z\}, instrukcemi \{L, R\} a stavy \{stavA, stavA-L, stavA-R .. stavZ, stavZ-L, stavZ-R\}.  

Neboli pronasobime stavy a instrukce, tim se pocet stavu ztrojnasobi a stav si bude pamatovat i nasledujici instrukci.  
Pokud vydime stav bez instrukce tak prepiseme znak na pasce a nastavime stav s novou instrukci (stavA $\rightarrow$ stavA-L), jenz bychom normalne vyzadovali.
Pokud vydime stav s instrukci, tak instrukci vykoname a stav zmenime na totozny bez instrukce (stavA-L $\rightarrow$ stavA).

\subsection*{10}
Mejme turinguv stroj $M$ s abecedou \{a .. z\}, instrukcemi \{L, R\} a stavy \{stavA..stavZ\}.

Pak $M'$ bude levy resetovaci turinguv stroj s abecedou
\{a, a$\alpha$, a$\beta$ .. z, z$\alpha$, z$\beta$ \}
(kde \# je placeholder pro symbol bez $\alpha$ nebo $\beta$. $\alpha$ bude zdrojova adresa a $\beta$ bude iterator),
Instrukcemi \{R, 2R, RESET\} a stavy \{stavA .. stavZ\} $\times$ \{\_, searching, found, shifting\}.

Instrukce R bude stejna jako v puvodnim TM. Instrukce L bude nahrazena touto posloupnosti funkci:
\begin{itemize}
    \item L $\rightarrow$ zmen aktualni symbol na verzi s $\alpha$ (\# $\rightarrow$ \#$\alpha$)
    \item proved instrukci \textbf{RESET}
    \item nastav symbol s $\beta$ (\# $\rightarrow$ \#$\beta$)
    \item proved instrukci \textbf{R}
    \item nastav symbol s $\beta$ (\# $\rightarrow$ \#$\beta$)
    \item proved instrukci \textbf{RESET}
    \item zmen stav na \textit{searching}
\end{itemize}
A pridame dalsi prechodove funkce:
\begin{enumerate}
    \item \textit{searching} \& \# $\rightarrow$ proved instrukci \textbf{R}\\
    (pokud nastane stav \textit{searching} a znak bude \#, tak proved instrukci \textbf{R})
    \item \textit{searching} \& \#$\beta$ $\rightarrow$ odeber ze symbolu $\beta$, proved instrukci \textbf{2R} a zmen stav na \textit{shifting}
    \item \textit{shifting} \& \# $\rightarrow$ pridej k symbolu $\beta$ a proved \textbf{RESET}
    \item \textit{shifting} \& \#$\alpha$ $\rightarrow$ zmen stav na \textit{found}, odeber ze symbolu $\alpha$ a proved \textbf{RESET}
    \item \textit{found} \& \# $\rightarrow$ proved instrukci \textbf{R}
    \item \textit{found} \& \#$\beta$ $\rightarrow$ odeber ze symbolu $\beta$, (konec posunu vlevo,) pokracuj s puvodnim stavem \_
\end{enumerate}

Jelikoz se L da stabilne pouzit pouze od 3. indexu, tak se pridaji jeste specialni funkce na L, ktere je na druhem indexu.
A pro L na prvnim indexu (cislovano od jedna) chovani TM nenadefinujeme, protoze doleva jiz jit nelze. 

\end{document}