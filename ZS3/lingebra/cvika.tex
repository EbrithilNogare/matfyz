\documentclass[a4paper]{article}
\usepackage[utf8]{inputenc}   % pro unicode UTF-8
\usepackage[czech]{babel} %jazyk dokumentu
\usepackage{listings}
\usepackage{color}
\usepackage[T1]{fontenc}
\usepackage{amssymb}
\usepackage{hyperref}
\usepackage{listingsutf8}
\usepackage{graphicx}
\usepackage{amsmath}
\usepackage{lipsum}


\newcommand\aug{\fboxsep=-\fboxrule\!\!\!\fbox{\strut}\!\!\!}

\graphicspath{ {/} }

%%%%%%%%%%%%%%%%%%%%%%%%%%%%%%%%%%%%%%%%%%%%%%%%%%%%%%%%%%%%%

\begin{document}
\section{jak se mela resit minula pisemka}
$$
\begin{bmatrix}
    1 & 2 & 03 & 4 \\
	2 & p & 4  & 1  \\
    3 & 4 & 1 &2   \\
    4 &1 &2 & p 
\end{bmatrix}
$$
REF kam az to pujde
$$
\begin{bmatrix}
    1 & 2 & 03 & 4 \\
	0 & p+1 & 3  & 3  \\
    0 & 3 & 2 &0   \\
    0 &3 &0 & p+4 
\end{bmatrix}
$$
vybereme si jen radky a sloupce co poterbujeme\\
protoze zbytek uz ma pivoty definovane
$$
\begin{bmatrix}
    p+4 & 3 \\
	3 & p+4
\end{bmatrix}
$$
Zjednodusime a vyzkousime vse v $Z^5$
$$
\begin{bmatrix}
    1 & 2p+3 \\
	0 & 2p+1-p(2p+3)
\end{bmatrix}
$$
\section{jsou linearne nezavisle?}
v $\mathbb{R}^4: \\
x_1=(1,2,0,0)^T \\
x_2=(2,1,1,3)^T \\
x_3=(0,1,0,1)^T
$
\\
matice pro vypocet:
$$
\begin{bmatrix}
    1 & 2 & 0 \\
    2 & 1 & 1 \\
    0 & 1 & 0 \\
    0 & 3 & 1 
\end{bmatrix}
-REF->
\begin{bmatrix}
    1 & 2 & 0 \\
    0 & 1 & 0 \\
    0 & 1 & 1 \\
    0 & 0 & 0 
\end{bmatrix}
$$

tudiz ma dimenzi 3 a tudiz jsou ty vektory linearne nezavisle
\\
jaky vektor jeste muzeme prdat aby byly stale linearne nezavisle?

$$
\begin{bmatrix}
    1 & 2 & 0 & a \\
    2 & 1 & 1 & b \\
    0 & 1 & 0 & c \\
    0 & 3 & 1 & d 
\end{bmatrix}
~
\begin{bmatrix}
    1 & 2 & 0 & a \\
    0 & 1 & 0 & c \\
    0 & 0 & 1 & b-2a+3c \\
    0 & 0 & 1 & d-3c 
\end{bmatrix}
~
\begin{bmatrix}
    1 & 2 & 0 & a \\
    0 & 1 & 0 & c \\
    0 & 0 & 1 & b-2a+3c \\
    0 & 0 & 0 & d-3c-(b-2a+3c) 
\end{bmatrix}
$$
vysledek:
$$ 2a - b - 6c + d \neq 0 $$
\\


\section{jak vypocist souradnice s jinou bazi}
zadani: v $\mathbb{R}^4:\\
B=((1,-3,7,2)^T,(3,2,1,-4)^T,(0,-1, 4,-3)^T,(-2, 4, -3, 0)^T)\\
x_1=(2, 2, 9, -5)^T \\
x_2=(-7, 2, 9, -8)^T \\
x_3=(-2, 4, -3, 0)^T \\
$

dosazeni do matice:
$$
\begin{bmatrix}
    1 & 3 & 0 & -2 &\aug& 2&-7 &-2 \\
    -3 & 2 &-1 &4  &\aug& 2&2 &4 \\
    7 & 1 & 4 &-3  &\aug& 9& 9 & -3 \\
    2 & -4& -3 & 0 &\aug& -5& -8 &0
\end{bmatrix}
RREF
\begin{bmatrix}
    1 & 0 & 0 & 0 &\aug& 1 & 0 & 2 \\
    0 & 1 & 0 & 0 &\aug& 1 &-1 &-4 \\
    0 & 0 & 1 & 0 &\aug& 1 & 4 & 0 \\
    0 & 0 & 0 & 1 &\aug& 1 & 2 & -7
\end{bmatrix}
$$
tudiz nam vzniknou nove souradnice:\\
$[x_1]_B = (1, 1, 1, 1)^T  $\\
$[x_2]_B = (0, -1, 4, 2)^T $\\
$[x_3]_B = (2, -4, 0, -7)^T$

\section{}

















\end{document}