\documentclass[a4paper]{article}
\usepackage[utf8]{inputenc}   % pro unicode UTF-8
\usepackage[czech]{babel} %jazyk dokumentu
\usepackage{listings}
\usepackage{color}
\usepackage[T1]{fontenc}
\usepackage{amssymb}
\usepackage{hyperref}
\usepackage{listingsutf8}
\usepackage{graphicx}
\usepackage{amsmath}
\usepackage{lipsum}
\usepackage[margin={1cm,1cm}]{geometry}


\newcommand\aug{\fboxsep=-\fboxrule\!\!\!\fbox{\strut}\!\!\!}

\graphicspath{ {/} }

%%%%%%%%%%%%%%%%%%%%%%%%%%%%%%%%%%%%%%%%%%%%%%%%%%%%%%%%%%%%%

\begin{document}
\section{22.11.}
\subsection{jak se mela resit minula pisemka}
$$
\begin{bmatrix}
	1 & 2 & 03 & 4 \\
	2 & p & 4  & 1  \\
	3 & 4 & 1 &2   \\
	4 &1 &2 & p 
\end{bmatrix}
$$
REF kam az to pujde
$$
\begin{bmatrix}
	1 & 2 & 03 & 4 \\
	0 & p+1 & 3  & 3  \\
	0 & 3 & 2 &0   \\
	0 &3 &0 & p+4 
\end{bmatrix}
$$
vybereme si jen radky a sloupce co poterbujeme\\
protoze zbytek uz ma pivoty definovane
$$
\begin{bmatrix}
	p+4 & 3 \\
	3 & p+4
\end{bmatrix}
$$
Zjednodusime a vyzkousime vse v $Z^5$
$$
\begin{bmatrix}
	1 & 2p+3 \\
	0 & 2p+1-p(2p+3)
\end{bmatrix}
$$
\subsection{jsou linearne nezavisle?}
v $\mathbb{R}^4: \\
x_1=(1,2,0,0)^T \\
x_2=(2,1,1,3)^T \\
x_3=(0,1,0,1)^T
$
\\
matice pro vypocet:
$$
\begin{bmatrix}
	1 & 2 & 0 \\
	2 & 1 & 1 \\
	0 & 1 & 0 \\
	0 & 3 & 1 
\end{bmatrix}
-REF->
\begin{bmatrix}
	1 & 2 & 0 \\
	0 & 1 & 0 \\
	0 & 1 & 1 \\
	0 & 0 & 0 
\end{bmatrix}
$$

tudiz ma dimenzi 3 a tudiz jsou ty vektory linearne nezavisle
\\
jaky vektor jeste muzeme prdat aby byly stale linearne nezavisle?

$$
\begin{bmatrix}
	1 & 2 & 0 & a \\
	2 & 1 & 1 & b \\
	0 & 1 & 0 & c \\
	0 & 3 & 1 & d 
\end{bmatrix}
~
\begin{bmatrix}
	1 & 2 & 0 & a \\
	0 & 1 & 0 & c \\
	0 & 0 & 1 & b-2a+3c \\
	0 & 0 & 1 & d-3c 
\end{bmatrix}
~
\begin{bmatrix}
	1 & 2 & 0 & a \\
	0 & 1 & 0 & c \\
	0 & 0 & 1 & b-2a+3c \\
	0 & 0 & 0 & d-3c-(b-2a+3c) 
\end{bmatrix}
$$
vysledek:
$$ 2a - b - 6c + d \neq 0 $$
\\


\subsection{jak vypocist souradnice s jinou bazi}
zadani: v $\mathbb{R}^4:\\
B=((1,-3,7,2)^T,(3,2,1,-4)^T,(0,-1, 4,-3)^T,(-2, 4, -3, 0)^T)\\
x_1=(2, 2, 9, -5)^T \\
x_2=(-7, 2, 9, -8)^T \\
x_3=(-2, 4, -3, 0)^T \\
$

dosazeni do matice:
$$
\begin{bmatrix}
	1 & 3 & 0 & -2 &\aug& 2&-7 &-2 \\
	-3 & 2 &-1 &4  &\aug& 2&2 &4 \\
	7 & 1 & 4 &-3  &\aug& 9& 9 & -3 \\
	2 & -4& -3 & 0 &\aug& -5& -8 &0
\end{bmatrix}
RREF
\begin{bmatrix}
	1 & 0 & 0 & 0 &\aug& 1 & 0 & 2 \\
	0 & 1 & 0 & 0 &\aug& 1 &-1 &-4 \\
	0 & 0 & 1 & 0 &\aug& 1 & 4 & 0 \\
	0 & 0 & 0 & 1 &\aug& 1 & 2 & -7
\end{bmatrix}
$$
tudiz nam vzniknou nove souradnice:\\
$[x_1]_B = (1, 1, 1, 1)^T  $\\
$[x_2]_B = (0, -1, 4, 2)^T $\\
$[x_3]_B = (2, -4, 0, -7)^T$




\section{29.11.}
\subsection{hledani bazi}

V $\mathbb{R}^5$ mejme vektorove podprostory $U, V$ definovane\\
$U=span({(2,1,-2,1,-1)^T,(2, 4, -2, -1, 1)^T,(4, 1, -4, 3, -3)^T})$
\\
$V=span({
	(0,1,1,0,-1)^T,
	(1, 2, -1, -2, 0)^T,
	(1, ,1 -2, -2, 1)^T,
	(1, 4, 0, -3, 0)^T,
	(2, 6, -1, -5, 0)^T
})$
\\
Najdete baze podprostoru $U, V, U \cap V$ a $U+V$

$
\begin{bmatrix}
	2 & 2 & 4 \\
	1 & 4 & 1 \\
	-2 & -2 & -4 \\
	1 & -1 & 3 \\
	-1 & 1 & -3 
\end{bmatrix}
REF
\begin{bmatrix}
	1 & 0 & 0 \\
	0 & 1 & 0 \\
	0 & 0 & 1 \\
	0 & 0 & 0 \\
	0 & 0 & 0
\end{bmatrix}
$\\
tudiz dimenze je tri ( $dim(U)=3$ ) \\
pozorovani: nektere radky jsou stejne (jen jine znamenko), ty se daji lehce odstranit
\\

\noindent v $\mathbb{R}^5$: \\
$dim(U)=dim(V)=3$ \\
$dim(U+V)=5$ \\
pro sjednoceni i soucet se udela $REF$ vseho v jedne matici, s tim ze se tam nemusi davat duplicitni \uv{sloupecky}
\\

\subsection{baze (pokracovani z minula)}
$B=(x^4+x^3, x^3+x^2, x^2+x, x+1, 1+x^4)$
\\
$f(x)=4x^4+4x^3+x+3$
\\
$[f(x)]_B = (a,b,c,d,e)^T$
\\
$
\begin{bmatrix}
	1 & 0 & 0 & 0 & 1 &\aug& 4 \\
	1 & 1 & 0 & 0 & 0 &\aug& 4 \\
	0 & 1 & 1 & 0 & 0 &\aug& 0 \\
	0 & 0 & 1 & 1 & 0 &\aug& 1 \\
	0 & 0 & 0 & 1 & 1 &\aug& 3
\end{bmatrix}
RREF
\begin{bmatrix}
	1 & 0 & 0 & 0 & 0 &\aug& 3 \\
	0 & 1 & 0 & 0 & 0 &\aug& 1 \\
	0 & 0 & 1 & 0 & 0 &\aug& -1 \\
	0 & 0 & 0 & 1 & 0 &\aug& 2 \\
	0 & 0 & 0 & 0 & 1 &\aug& 1
\end{bmatrix}
=(3,1,-1,2,1)^T
$


\subsection{co se bude dit, kdyz u prvniho prikladu zanedbame transpozici}
dimenze sice zustane, ale ztratime prehled o tom ktery vektor je ktery radek,
coz se nam pri sloupcovem zapisu nestane
\\
v obou pripadech to ulohu nejak resi, ale vysledek se pak musi \uv{spravne precist}
\\



\section{20.12.}
\subsection{reseni pisemky}
$
f((0,0,1)^T)=(1,2,3)^T\\
f((0,1,1)^T)=(1,0,0)^T\\
f((1,1,1)^T)=(-1,2,3)^T\\
$
\\mezivypocet:\\
odectenim privniho vektoru od druheho dostaneme druhy vektor kan. baze\\
podobne i pro treti vektor...\\
$
f((0,1,0)^T) = (0,-2-,-3)^T\\
f((1,0,0)^T) = (-2,2,3)^T\\
$
\\vysledek:\\
$
\begin{bmatrix}
	-2 & 0 & 1 \\
	2 & -2 & 2 \\
	3 & -3 & 3 
\end{bmatrix}
$

\subsection{priklad na linearni zobrazeni}
$f:Z^3_5 \rightarrow Z^3_5$\\
$
f((2,4,1)^T) = (2,1,2)^T\\
f((2,3,4)^T) = (0,4,1)^T\\
f((3,0,1)^T) = (4,4,1)^T\\
$\\
$
\begin{bmatrix}
	-2 & 0 & 1 \\
	2 & -2 & 2 \\
	3 & -3 & 3 
\end{bmatrix}
= {}_{kan}[f]_B
$\\to co mame krat matice prechodu = to co hledame\\ 
$
{}_{kan}[f]_B {}_B[id]_{kan} = 
{}_{kan}[f]_B {}_{kan}[id]_B^{-1} = 
{}_{kan}[f]_{kan}
$\\\textbf{veta}:\\
$
{}_{B3}[gf]_{B_1} =
{}_{B_3}[g]_{B2} {}_{B_2}[f]_{B1} 
$\\vysledek:\\
$
{}_k[f]_k = 
\begin{bmatrix}
	4 & 3 & 2 \\
	2 & 1 & 3 \\
	0 & 4 & 1 
\end{bmatrix}
$\\\textbf{pozorovani}:\\
$
{}_k[f]_k * [A] = [B] \rightarrow {}_k[f]_k = [b] * [A]^{-1} 
$ , kde $[A]$ a $[B]$ jsou matice

\subsection{dalsi priklad na linearni zobrazeni}
$
Z^4_5 : (1,2,0,1)^T, (4,1,3,1)^T, (3,1,3,4)^T, (2,0,2,2)^T\\
		(1,2,3,1)^T, (4,4,1,1)^T, (2,0,2,1)^T, (3,1,4,0)^T
$\\\\
$
{}_{B2}[id]_{B1} = {}_{B2}[id]_k {}_k[id]_{B1}\\
= ({}_k[id]_{B2})^{-1} {}_k[id]_{B1}
$

\subsection{linearni zobrazeni v komplexnich cislech}
$f:\mathbb{C} \rightarrow \mathbb{C}$\\
a) $f_1(a+bi) = a - bi$ \dots neplati definice linearniho zobrazeni tudiz neni linearni zobrazeni\\
b) $f_2(a+bi) = -b + ai$\\














\end{document}