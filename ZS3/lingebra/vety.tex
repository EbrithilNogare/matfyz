\documentclass[a4paper]{article}
\usepackage[utf8]{inputenc}   % pro unicode UTF-8
\usepackage[czech]{babel} %jazyk dokumentu
\usepackage{listings}
\usepackage{color}
\usepackage[T1]{fontenc}
\usepackage{amssymb}
\usepackage{hyperref}
\usepackage{listingsutf8}
\usepackage{graphicx}
\usepackage{amsmath}
\usepackage[margin={1cm,2cm}]{geometry}


\newcommand\aug{\fboxsep=-\fboxrule\!\!\!\fbox{\strut}\!\!\!}

\newcommand{\asubsection}[3]{
	\setcounter{section}{#1}
	\setcounter{subsection}{#2}
	\addtocounter{subsection}{-1}
	\subsection{#3}~
}

\graphicspath{ {/} }

%%%%%%%%%%%%%%%%%%%%%%%%%%%%%%%%%%%%%%%%%%%%%%%%%%%%%%%%%%%%%

\begin{document}

\section*{Definice}

\asubsection{2}{2}{Matice}
Rálná matice typu $m \times n$ je obdélníkové schema (tabulka)


\asubsection{2}{3}{Vektor}
Reálný n-rozměrný aritmetický sloupcový vektor je matice typu  $m \times 1$


\asubsection{2}{4}{* notace}
i-tý řádek matice $A$ se značí: $A_{i*} = (a_{i1}, a_{i2}, . . . , a_{in})$


\asubsection{2}{5}{Soustava lineárních rovnic}


\asubsection{2}{6}{Matice soustavy}


\asubsection{2}{8}{Elementární řádkové úpravy}


\asubsection{2}{12}{Odstupňovaný tvar matice}


\asubsection{2}{13}{Hodnost matice}


\asubsection{2}{18}{Redukovaný odstupňovaný tvar matice}


\asubsection{3}{1}{Rovnost}


\asubsection{3}{2}{Součet}


\asubsection{3}{3}{Násobek}


\asubsection{3}{7}{Součin}


\asubsection{3}{11}{Transpozice}


\asubsection{3}{14}{Symetrická matice}


\asubsection{3}{23}{Regulární matice}


\asubsection{3}{30}{Inverzní matice}


\asubsection{4}{1}{Grupa}


\asubsection{4}{5}{Podgrupa}


\asubsection{4}{8}{Permutace}


\asubsection{4}{9}{Inverzní permutace}


\asubsection{4}{1}{Skládání permutací}


\asubsection{4}{13}{Znaménko permutace}


\asubsection{4}{22}{Těleso}


\asubsection{4}{35}{Charakteristika tělesa}


\asubsection{5}{1}{Vektorový prostor}


\asubsection{5}{4}{Podprostor}


\asubsection{5}{8}{Lineární obal}


\asubsection{5}{11}{Lineární kombinace}


\asubsection{5}{21}{Lineární nezávislost}


\asubsection{5}{22}{Lineární nezávislost nekonečné množiny}


\asubsection{5}{29}{Báze}


\asubsection{5}{32}{Souřadnice}


\asubsection{5}{42}{Dimenze}


\asubsection{5}{49}{Spojení podprostorů}


\asubsection{5}{55}{Maticové prostory}


\asubsection{6}{1}{Lineární zobrazení}


\asubsection{6}{6}{Obraz a jádro}


\asubsection{6}{14}{Matice lineárního zobrazení}


\asubsection{6}{20}{Matice přechodu}


\asubsection{6}{29}{Isomorfismus}


\asubsection{6}{41}{Prostor lineárních zobrazení}


\asubsection{7}{1}{Afinní podprostor}


\asubsection{7}{7}{Dimenze afinního podprostoru}


\asubsection{7}{10}{Afinní nezávislost}



\newpage
\section*{Věty}

\asubsection{1}{1}{Základní věta algebry}


\asubsection{3}{28}{o regularni matici}


\asubsection{3}{31}{O existenci inverzní matice}


\asubsection{3}{33}{Jedna rovnost stačí}


\asubsection{3}{34}{Výpočet inverzní matice}


\asubsection{3}{37}{Soustava rovnic a inverzní matice}


\asubsection{3}{41}{Shermanova–Morrisonova formule}


\asubsection{3}{43}{Jednoznačnost RREF}


\asubsection{4}{15}{O znaménku složení permutace a transpozice}


\asubsection{4}{16}{Každou permutaci lze rozložit na složení transpozic}


\asubsection{4}{27}{$Z_n$ je těleso právě tehdy, když n je prvočíslo}


\asubsection{4}{33}{O velikosti konečných těle}


\asubsection{4}{38}{Malá Fermatova věta}


\asubsection{5}{15}{o vektorovem prostoru a obalu}


\asubsection{5}{26}{vektor nad T ...}


\asubsection{5}{31}{o bazi}


\asubsection{5}{38}{O existenci báze}


\asubsection{5}{40}{Steinitzova věta o výměně}


\asubsection{5}{44}{Vztah počtu prvků systému k dimenzi}


\asubsection{5}{45}{Rozšíření lineárně nezávislého systému na bázi}


\asubsection{5}{46}{Dimenze podprostoru}


\asubsection{5}{50}{Spojení podprostorů}


\asubsection{5}{52}{Dimenze spojení a průniku}


\asubsection{5}{62}{Maticové prostory a RREF}


\asubsection{5}{63}{Pro každou matici $A \in T m\times n$ platí rank(A) = rank($A^T$)}


\asubsection{5}{66}{O dimenzi jádra a hodnosti matice}


\asubsection{6}{10}{Prosté lineární zobrazení}


\asubsection{6}{12}{Lineární zobrazení a jednoznačnost vzhledem k obrazům báze}


\asubsection{6}{16}{Maticová reprezentace lineárního zobrazení}


\asubsection{6}{18}{Jednoznačnost matice lineárního zobrazení}


\asubsection{6}{24}{Matice složeného lineárního zobrazeni}


\asubsection{6}{35}{Isomorfismus n-dimenzionálních prosto}


\asubsection{6}{37}{O dimenzi jádra a obrazu}


\asubsection{7}{4}{Charakterizace afinního podprostoru}


\asubsection{7}{5}{o Množina řešení soustavy rovni}



\end{document}