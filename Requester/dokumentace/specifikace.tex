\documentclass[a4paper]{article}
\usepackage[utf8]{inputenc}
\usepackage{amsmath}
\usepackage{amsfonts}
\usepackage{amssymb}
\setlength{\parindent}{0in}
\usepackage{fancyhdr}
\usepackage[sfdefault]{roboto}
\usepackage{roboto-mono}

\begin{document}

\pagestyle{fancy}
\rhead{David Nápravník}

{
    \huge{
		\textsf{
			\centerline{
				\textbf{
                    specifikace k projektu Requester
                }
			}
		}
    }
}
    
\section{motivace}
Program ktery dokaze plne modifikovat request http protokolu.

\section{detailni specifikace}
Program bude umet odeslat POST nebo GET request pres http.\\
V requestu bude jednoduse mozne menit veskere vlastnosti vcetne hlavicky.\\
Response se bude parsovat a bude jej mozne dale zpracovavat.

\section{Prostredi}
Program bude mozne spustit jako aplikaci vcetne GUI, nebo
jako headless, kdy dostane instrukce co ma delat a to bud pres
config, nebo parametry programu.

\section{Vyuziti}
\begin{itemize}
    \item automaticke stahovani souboru z webu
    \item automaticke testovani webovych aplikaci
    \item bruteforce webovych aplikaci
    \item ziskavani syntetickych dat pres http
\end{itemize}

\section{Technologie}
\begin{itemize}
    \item HTTP Client (.Net \textasciicircum 4.5)
    \item Material design pro GUI
    \item AJAX
    \begin{itemize}
        \item multivlaknove rozesilani/zpracovavani pozadavku
        \item Lazy vyhodnocovani
    \end{itemize}
\end{itemize}

\end{document}