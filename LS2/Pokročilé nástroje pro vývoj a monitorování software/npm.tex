\documentclass{beamer}

\mode<presentation> {\usetheme{Frankfurt}}

\usepackage{graphicx} % Allows including images
\usepackage{booktabs} % Allows the use of \toprule, \midrule and \bottomrule in tables
\graphicspath{ {./images/} }

\setbeamertemplate{navigation symbols}{}

%----------------------------------------------------------------------------------------
%	TITLE PAGE
%----------------------------------------------------------------------------------------

\title[Short title]{
	\includegraphics[scale=0.4]{npm_logo}	
} % The short title appears at the bottom of every slide, the full title is only on the title page

\author{David Nápravník} % Your name
\institute[mff] % Your institution as it will appear on the bottom of every slide, may be shorthand to save space
{
Charles University \\ % Your institution for the title page
\medskip
\textit{ebrithil@nogare.cz} % Your email address
}
\date{\today} % Date, can be changed to a custom date

\begin{document}

\begin{frame}
\titlepage % Print the title page as the first slide
\end{frame}

%----------------------------------------------------------------------------------------
%	PRESENTATION SLIDES
%----------------------------------------------------------------------------------------

\section{Introduction}

\begin{frame}
\frametitle{Motivation}
\begin{itemize}
\item separate large web projects into tiny pieces
\item reuse these pieces in another project
\item resolve dependency
\item fast import usefull packages of working code
\item online database of open-source projects
\end{itemize}


\end{frame}

\begin{frame}
\frametitle{What it is}
\begin{itemize}
\item npm is the world’s largest software registry.
\item Open source developers from every continent use npm to share
and borrow packages, and many organizations use npm
to manage private development as well.
\end{itemize}
\end{frame}

\begin{frame}
\frametitle{Installation}
\begin{block}{Windows / mac}
\url{https://nodejs.org/en/download/current/}
\end{block}

\begin{block}{Linux}
sudo pacman -S nodejs npm\\
sudo apt-get install -y nodejs	
\end{block}

\begin{example}[verify]
npm -v\\
//0.6.9
\end{example}

\end{frame}


\section{Packages}

\begin{frame}
\frametitle{web database}
online database:  \textbf{www.npm.com}\\
contains documentation:
\begin{itemize}
\item readme (what that should do)
\item install (how to import it into your project)
\item usage (how to use it)
\end{itemize}
\end{frame}

\begin{frame}
\frametitle{package}
packages can be added by everybody
\begin{itemize}
\item bugs (unkempt packages)
\item malicious code
\end{itemize}
git repositories
\begin{itemize}
\item versions
\item code checkout
\item another examples
\end{itemize}
credibility
\begin{itemize}
\item report button
\item weekly downloads
\item version
\item last publish
\end{itemize}
\end{frame}

\begin{frame}
\frametitle{dependencies}
production
\begin{itemize}
\item for project funkcionality
\item eg. react
\end{itemize}
dev
\begin{itemize}
\item for purpose of development
\item eg. eslint
\end{itemize}
\end{frame}

\begin{frame}
\frametitle{upload}
version
login
\end{frame}

\section{Configs}

\begin{frame}
\frametitle{package.json}
packages
run
version
\end{frame}

\begin{frame}
\frametitle{}

\end{frame}

\begin{frame}
\frametitle{}

\end{frame}

\begin{frame}
\frametitle{}

\end{frame}


\section{Usage}

\begin{frame}
\frametitle{init}

\end{frame}

\begin{frame}
\frametitle{console}

\end{frame}

\begin{frame}
\frametitle{webpack}

\end{frame}

\begin{frame}
\frametitle{adding package}
uniqs
\end{frame}




















%------------------------------------------------
\section{example pages}
%------------------------------------------------


\begin{frame}

\end{frame}
\begin{frame}

\end{frame}
\begin{frame}
\frametitle{Paragraphs of Text}
Sed iaculis dapibus gravida. Morbi sed tortor erat, nec interdum arcu. Sed id lorem lectus. Quisque viverra augue id sem ornare non aliquam nibh tristique. Aenean in ligula nisl. Nulla sed tellus ipsum. Donec vestibulum ligula non lorem vulputate fermentum accumsan neque mollis.\\~\\

Sed diam enim, sagittis nec condimentum sit amet, ullamcorper sit amet libero. Aliquam vel dui orci, a porta odio. Nullam id suscipit ipsum. Aenean lobortis commodo sem, ut commodo leo gravida vitae. Pellentesque vehicula ante iaculis arcu pretium rutrum eget sit amet purus. Integer ornare nulla quis neque ultrices lobortis. Vestibulum ultrices tincidunt libero, quis commodo erat ullamcorper id.
\end{frame}

%------------------------------------------------

\begin{frame}
\frametitle{Bullet Points}
\begin{itemize}
\item Lorem ipsum dolor sit amet, consectetur adipiscing elit
\item Aliquam blandit faucibus nisi, sit amet dapibus enim tempus eu
\item Nulla commodo, erat quis gravida posuere, elit lacus lobortis est, quis porttitor odio mauris at libero
\item Nam cursus est eget velit posuere pellentesque
\item Vestibulum faucibus velit a augue condimentum quis convallis nulla gravida
\end{itemize}
\end{frame}

%------------------------------------------------

\begin{frame}
\frametitle{Blocks of Highlighted Text}
\begin{block}{Block 1}
Lorem ipsum dolor sit amet, consectetur adipiscing elit. Integer lectus nisl, ultricies in feugiat rutrum, porttitor sit amet augue. Aliquam ut tortor mauris. Sed volutpat ante purus, quis accumsan dolor.
\end{block}

\begin{block}{Block 2}
Pellentesque sed tellus purus. Class aptent taciti sociosqu ad litora torquent per conubia nostra, per inceptos himenaeos. Vestibulum quis magna at risus dictum tempor eu vitae velit.
\end{block}

\begin{block}{Block 3}
Suspendisse tincidunt sagittis gravida. Curabitur condimentum, enim sed venenatis rutrum, ipsum neque consectetur orci, sed blandit justo nisi ac lacus.
\end{block}
\end{frame}

%------------------------------------------------

\begin{frame}
\frametitle{Multiple Columns}
\begin{columns}[c] % The "c" option specifies centered vertical alignment while the "t" option is used for top vertical alignment

\column{.45\textwidth} % Left column and width
\textbf{Heading}
\begin{enumerate}
\item Statement
\item Explanation
\item Example
\end{enumerate}

\column{.5\textwidth} % Right column and width
Lorem ipsum dolor sit amet, consectetur adipiscing elit. Integer lectus nisl, ultricies in feugiat rutrum, porttitor sit amet augue. Aliquam ut tortor mauris. Sed volutpat ante purus, quis accumsan dolor.

\end{columns}
\end{frame}

\begin{frame}
\frametitle{Table}
\begin{table}
\begin{tabular}{l l l}
\toprule
\textbf{Treatments} & \textbf{Response 1} & \textbf{Response 2}\\
\midrule
Treatment 1 & 0.0003262 & 0.562 \\
Treatment 2 & 0.0015681 & 0.910 \\
Treatment 3 & 0.0009271 & 0.296 \\
\bottomrule
\end{tabular}
\caption{Table caption}
\end{table}
\end{frame}

%------------------------------------------------

\begin{frame}
\frametitle{Theorem}
\begin{theorem}[Mass--energy equivalence]
$E = mc^2$
\end{theorem}
\end{frame}

%------------------------------------------------

\begin{frame}[fragile] % Need to use the fragile option when verbatim is used in the slide
\frametitle{Verbatim}
\begin{example}[Theorem Slide Code]
\begin{verbatim}
\begin{frame}
\frametitle{Theorem}
\begin{theorem}[Mass--energy equivalence]
$E = mc^2$
\end{theorem}
\end{frame}\end{verbatim}
\end{example}
\end{frame}

%------------------------------------------------

\begin{frame}
\frametitle{Figure}
Uncomment the code on this slide to include your own image from the same directory as the template .TeX file.
%\begin{figure}
%\includegraphics[width=0.8\linewidth]{test}
%\end{figure}
\end{frame}

%------------------------------------------------

\begin{frame}[fragile] % Need to use the fragile option when verbatim is used in the slide
\frametitle{Citation}
An example of the \verb|\cite| command to cite within the presentation:\\~

This statement requires citation \cite{p1}.
\end{frame}

%------------------------------------------------

\begin{frame}
\frametitle{References}
\footnotesize{
\begin{thebibliography}{99} % Beamer does not support BibTeX so references must be inserted manually as below
\bibitem[Smith, 2012]{p1} John Smith (2012)
\newblock Title of the publication
\newblock \emph{Journal Name} 12(3), 45 -- 678.
\end{thebibliography}
}
\end{frame}

%------------------------------------------------

\begin{frame}
\Huge{\centerline{The End}}
\end{frame}

%----------------------------------------------------------------------------------------

\end{document} %