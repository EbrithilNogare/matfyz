\documentclass{beamer}

\mode<presentation> {\usetheme{Frankfurt}}

\usepackage{graphicx} % Allows including images
\usepackage{booktabs} % Allows the use of \toprule, \midrule and \bottomrule in tables
\graphicspath{ {./images/} }

\setbeamertemplate{navigation symbols}{}

%----------------------------------------------------------------------------------------
%	TITLE PAGE
%----------------------------------------------------------------------------------------

\title[Short title]{
	\includegraphics[scale=0.4]{npm_logo}	
} % The short title appears at the bottom of every slide, the full title is only on the title page

\author{David Nápravník} % Your name
\institute[mff] % Your institution as it will appear on the bottom of every slide, may be shorthand to save space
{
Charles University \\ % Your institution for the title page
\medskip
\textit{ebrithil@nogare.cz} % Your email address
}
\date{\today} % Date, can be changed to a custom date

\begin{document}

\begin{frame}
\titlepage % Print the title page as the first slide
\end{frame}

%----------------------------------------------------------------------------------------
%	PRESENTATION SLIDES
%----------------------------------------------------------------------------------------

\section{Introduction}

\begin{frame}
\frametitle{Motivation}
\begin{itemize}
\item split large web projects into tiny pieces
\item reuse these pieces in another project
\item resolve dependency
\item fast import usefull packages of working code
\item online database of open-source projects
\end{itemize}


\end{frame}

\begin{frame}
\frametitle{What it is}
\begin{itemize}
\item npm is the world’s largest software registry.
\item Open source developers from every continent use npm to share
and borrow packages, and many organizations use npm
to manage private development as well.
\end{itemize}
\end{frame}

\begin{frame}
\frametitle{Installation}
\begin{block}{Windows / mac}
\url{https://nodejs.org/en/download/current/}
\end{block}

\begin{block}{Linux}
sudo pacman -S nodejs npm\\
sudo apt-get install -y nodejs	
\end{block}

\begin{example}[verify]
npm -v\\
//0.6.9
\end{example}

\end{frame}


\section{Packages}

\begin{frame}
\frametitle{web database}
online database:  \textbf{www.npmjs.com}\\
contains documentation:
\begin{itemize}
\item readme (what that should do)
\item install (how to import it into your project)
\item usage (how to use it)
\end{itemize}
\end{frame}

\begin{frame}
\frametitle{package}
packages can be added by everybody
\begin{itemize}
\item bugs (unkempt packages)
\item malicious code
\end{itemize}
git repositories
\begin{itemize}
\item versions
\item code checkout
\item another examples
\end{itemize}
credibility
\begin{itemize}
\item report button
\item weekly downloads
\item version
\item last publish
\end{itemize}
\end{frame}

\begin{frame}
\frametitle{dependencies}
real example of reusable code \\
\textbf{types:}\\
production
\begin{itemize}
\item for project funkcionality
\item eg. react
\end{itemize}
dev
\begin{itemize}
\item for purpose of development
\item eg. eslint
\end{itemize}
\end{frame}


\section{Configs}

\begin{frame}
\frametitle{package.json}
\begin{itemize}
\item basic information about project
\begin{itemize}
\item name
\item version
\item repository
\end{itemize}
\item scripts
\item other configs
\item dependencies
\begin{itemize}
\item + devDependencies
\end{itemize}
\end{itemize}
\end{frame}

\begin{frame}
\frametitle{webpack.config.js}
\begin{itemize}
\item import packages
\item plugins
\item modules
\item build specification
\end{itemize}
\end{frame}

\begin{frame}
\frametitle{.ignores}
\begin{itemize}
\item .gitinore
\begin{itemize}
\item node\_modules
\end{itemize}
\item .npmignore
\begin{itemize}
\item tests
\item same as git
\end{itemize}
\end{itemize}
\end{frame}



\section{Usage}

\begin{frame}
\frametitle{init}
\begin{block}{init npm folder}
npm init
\end{block}
\begin{block}{copy default config}
\end{block}
\begin{block}{install packages}
npm install
\end{block}
\end{frame}

\begin{frame}
\frametitle{npm scripts}
\begin{block}{console}
node entryPoint.js
\end{block}
\begin{block}{webpack}
node run scriptName
\end{block}
\end{frame}

\begin{frame}
\frametitle{adding package}
\begin{block}{console}
node install uniqs
\end{block}
\begin{block}{package.json}
add uniqs into dependecies and run:\\
npm install
\end{block}

\end{frame}

\begin{frame}
\frametitle{upload}
\begin{itemize}
\item login to npm.com
\item create npm project
\item create git repository
\item fill important fields
\item npm publish
\end{itemize}
\end{frame}





\section{End}

\begin{frame}
\Huge{\centerline{Questions?}}
\end{frame}
\begin{frame}
\Huge{\centerline{The End}}
\end{frame}

\end{document} %