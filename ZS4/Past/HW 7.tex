\documentclass[a4paper]{article}
\usepackage[utf8]{inputenc}   % pro unicode UTF-8
\usepackage[czech]{babel} %jazyk dokumentu
\usepackage{listings}
\usepackage{color}
\usepackage[T1]{fontenc}
\usepackage{amssymb}
\usepackage{hyperref}
\usepackage{listingsutf8}
\usepackage{graphicx}
\usepackage{amsmath}
\usepackage{multicol}
\usepackage[margin={1cm,2cm}]{geometry}

\graphicspath{ {/} }

\def\doubleunderline#1{\underline{\underline{#1}}}

%%%%%%%%%%%%%%%%%%%%%%%%%%%%%%%%%%%%%%%%%%%%%%%%%%%%%%%%%%%%%

\begin{document}

\noindent
\textbf{Predmet: Pravděpodobnost a statistika 1}\\
\textbf{Ukol: 7.}\\
\textbf{Verze: 1.}\\
\textbf{Autor: David Napravnik}

\section*{4) Klacek}

\subsection*{a}

vysledkem je dvojnasobny obsah telesa (dvojnasobny, protoze bereme v uvahu jen pripad, kdy $x<y$)\\
$x>=0, x<=1, y>=0, y<=1, x<y, max(x,1-y,abs(x-y))<0.5$\\
coz je $1/4$


\subsection*{b}
podivame se pro ktere hodnoty x (bod zlomu), muzeme sestavit trojuhelnik\\
$\frac{2-\sqrt{2}}{2}<x<\frac{1}{2}$\\
$\frac{1}{2}-\frac{2-\sqrt{2}}{2} \simeq 0.207$


\subsection*{c}
to same jako uloha b), akorat dvojnasobek, protoze
mame dve nezavisle moznosti, ktery klacek napodruhe zlomit.\\
takze $\simeq 0.414$







\end{document}