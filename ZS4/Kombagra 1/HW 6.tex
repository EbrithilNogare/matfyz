\documentclass[a4paper]{article}
\usepackage[utf8]{inputenc}   % pro unicode UTF-8
\usepackage[czech]{babel} %jazyk dokumentu
\usepackage{listings}
\usepackage{color}
\usepackage[T1]{fontenc}
\usepackage{amssymb}
\usepackage{hyperref}
\usepackage{listingsutf8}
\usepackage{graphicx}
\usepackage{amsmath}
\usepackage{multicol}
\usepackage[margin={1cm,2cm}]{geometry}

\graphicspath{ {/} }

\setcounter{MaxMatrixCols}{40}
\def\doubleunderline#1{\underline{\underline{#1}}}

%%%%%%%%%%%%%%%%%%%%%%%%%%%%%%%%%%%%%%%%%%%%%%%%%%%%%%%%%%%%%

\begin{document}

\noindent
\textbf{Predmet: Kombinatorika a grafy 1}\\
\textbf{Ukol: 6.}\\
\textbf{Verze: 1.}\\
\textbf{Autor: David Napravnik}

\section*{Prvni ukol}
Generujici matice: $ G = 
\begin{bmatrix}
	1 & 0 & 0 & 0 & 1 & 1 & 0\\
	0 & 1 & 0 & 0 & 1 & 0 & 1\\
	0 & 0 & 1 & 0 & 0 & 1 & 1\\
	0 & 0 & 0 & 1 & 1 & 1 & 1\\
\end{bmatrix}
$\\
Parametry jsou $[4,7]$, dokaze opravit az 1 chybny bit.\\
Kontrolni matice: $ H = 
\begin{bmatrix}
	1 & 1 & 0 & 1 & 1 & 0 & 0\\
	1 & 0 & 1 & 1 & 0 & 1 & 0\\
	0 & 1 & 1 & 1 & 0 & 0 & 1\\
\end{bmatrix}
$\\
Odesilame zpravu $5 \sim 0b0101$\\
Enkodovana zprava = $
\begin{bmatrix}
	0 & 1 & 0 & 1\\
\end{bmatrix}
*
\begin{bmatrix}
	1 & 0 & 0 & 0 & 1 & 1 & 0\\
	0 & 1 & 0 & 0 & 1 & 0 & 1\\
	0 & 0 & 1 & 0 & 0 & 1 & 1\\
	0 & 0 & 0 & 1 & 1 & 1 & 1\\
\end{bmatrix}
=
\begin{bmatrix}
	0 & 1 & 0 & 1 & 0 & 1 & 0\\
\end{bmatrix}
$\\
Odesilame zpravu $9 \sim 0b1001$\\
Enkodovana zprava = $
\begin{bmatrix}
	1 & 0 & 0 & 1\\
\end{bmatrix}
*
\begin{bmatrix}
	1 & 0 & 0 & 0 & 1 & 1 & 0\\
	0 & 1 & 0 & 0 & 1 & 0 & 1\\
	0 & 0 & 1 & 0 & 0 & 1 & 1\\
	0 & 0 & 0 & 1 & 1 & 1 & 1\\
\end{bmatrix}
=
\begin{bmatrix}
	1 & 0 & 0 & 1 & 0 & 0 & 1\\
\end{bmatrix}
$\\
\subsection*{Prijimame zpravu 1100100}
Kontrola chyb: $
\begin{bmatrix}
	1 & 1 & 0 & 1 & 1 & 0 & 0\\
	1 & 0 & 1 & 1 & 0 & 1 & 0\\
	0 & 1 & 1 & 1 & 0 & 0 & 1\\
\end{bmatrix}
*
\begin{bmatrix}
	1 \\
	1 \\
	0 \\
	0 \\
	1 \\
	0 \\
	0 \\
\end{bmatrix}
=
[1,1,1]
$\\
jelikoz se chyba nasla na 4 pozici (4. slupec kontrolni matice),
tak puvodni zprava znela: $[1,1,0,1,1,0,0]$,
coz nam bez kontrolnich bitu dava cislo $0b1101 \sim 13$
\\
\subsection*{Prijimame zpravu 1111111}
Kontrola chyb: $
\begin{bmatrix}
	1 & 1 & 0 & 1 & 1 & 0 & 0\\
	1 & 0 & 1 & 1 & 0 & 1 & 0\\
	0 & 1 & 1 & 1 & 0 & 0 & 1\\
\end{bmatrix}
*
\begin{bmatrix}
	1 \\
	1 \\
	1 \\
	1 \\
	1 \\
	1 \\
	1 \\
\end{bmatrix}
=
[0,0,0]
$\\
jelikoz chyba nenasla (nulovy vektor),
tak puvodni zprava znela: $[1,1,1,1,1,1,1]$,
coz nam bez kontrolnich bitu dava cislo $0b1111 \sim 15$
\\
\subsection*{Prijimame zpravu 0110010}
Kontrola chyb: $
\begin{bmatrix}
	1 & 1 & 0 & 1 & 1 & 0 & 0\\
	1 & 0 & 1 & 1 & 0 & 1 & 0\\
	0 & 1 & 1 & 1 & 0 & 0 & 1\\
\end{bmatrix}
*
\begin{bmatrix}
	0 \\
	1 \\
	1 \\
	0 \\
	0 \\
	1 \\
	0 \\
\end{bmatrix}
=
[1,0,0]
$\\
jelikoz se chyba nasla na 5 pozici (5. slupec kontrolni matice),
tak puvodni zprava znela: $[0,1,1,0,1,1,0]$,
coz nam bez kontrolnich bitu dava cislo $0b0110 \sim 6$ \\

\section*{Druhy ukol}
Vezmeme jeden vrchol $v$ ten musi mit 16 hran k ostatnim bodum.\\
Jelikoz mame 3 barvy a 16 hran, tak podle holubnikoveho lematu, 
musi jedna barva byt zastoupena minimalne 6krat. (rekneme treba modra)\\
Vezmeme techto 6 hran stejne barvy a jejich vrcholy. Ty tvori $K_6$.\\
Rozebereme tri pripady co se deje uvnitr tohoto $K_6$ grafu:\\
\textbf{graf nema zadnou modrou hranu} \\
Tak je to kompletni graf na 6ti vrcholech obarveny 2mi barvami
a tudiz ma dva jednobarevne trojuhelniky.\\
\textbf{graf ma alespon jednu modrou hranu} \\
Tudiz existuje modra hrana mezi vrcholy $t$ a $u$ (hranu nazveme $tu$),
dale mame dve modre hrany $vt$ a $vu$ nebot kazdy vrchol z $K_6$ je napojen na $v$.
Tudiz dostavame trojuhelnik $tuv$ modre barvy.

\section*{Treti ukol}
R(3,4) = 9\\
Mejme graf $K_9$ obarveny modre a cervene.\\
Chceme dokazat, ze existuje modry ctverec nebo cerveny trojuhelnik.\\
Vezmeme vrchol $x$, ten muze mit maximalne 5 modrych hran,
protoze kdyby mel vice nez 5 modrych hran, tak by tvoril modry ctverec, nebo cerveny trojuhelnik.
A muze mit maximalne 3 cervene hrany, protoze kdyby jich mel vic, tak zase vznikne cerveny trojuhelnik nebo modry ctverec.\\
Jelikoz z vrcholu $x$ vychazi 8 hran, maximalne 5 je modrych a maximalne 3 cervene, tak ma prave 5 modrych a prave 3 cervene hrany.\\
Kazdy vrchol z $K_9$ tak musi mit prave 3 cervene a 5 modrych hran.
To ale neni mozne, protoze cervenych hran by bylo $3*9/2=13.5$, polovicni hrana nemuze existovat, proto musime cislo zaokrouhlit,
pokud cislo zaokrouhlime dolu, bude v grafu vrchol ktery ma 6 modrych hran,
pokud nahoru, tak bude existovat vrchol s 4mi cervenymi vrcholi. Obe tyto moznosti implikuji vznik cerveneho trojuhelniku nebo modreho ctverce.\\
Tudiz dokazano.








\end{document}