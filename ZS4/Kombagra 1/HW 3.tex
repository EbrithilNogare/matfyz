\documentclass[a4paper]{article}
\usepackage[utf8]{inputenc}   % pro unicode UTF-8
\usepackage[czech]{babel} %jazyk dokumentu
\usepackage{listings}
\usepackage{color}
\usepackage[T1]{fontenc}
\usepackage{amssymb}
\usepackage{hyperref}
\usepackage{listingsutf8}
\usepackage{graphicx}
\usepackage{amsmath}
\usepackage{multicol}
\usepackage[margin={1cm,2cm}]{geometry}

\graphicspath{ {/} }

\setcounter{MaxMatrixCols}{40}
\def\doubleunderline#1{\underline{\underline{#1}}}

%%%%%%%%%%%%%%%%%%%%%%%%%%%%%%%%%%%%%%%%%%%%%%%%%%%%%%%%%%%%%

\begin{document}

\noindent
\textbf{Predmet: Kombinatorika a grafy 1}\\
\textbf{Ukol: 3.}\\
\textbf{Verze: 1.}\\
\textbf{Autor: David Napravnik}

\section*{Prvni ukol}
Plati, protoze zpusob jakym vytvarime matice se da prevest na vytvareni konecne projektivni roviny radu $n-1$ a v  takove KPR primky predstavuji ortogonalni latinske ctverce




\section*{Druhy ukol}
$
\begin{matrix}
	2 & 3 & 4 & 0 & 1 \\
	3 & 4 & 0 & 1 & 2 \\
	4 & 0 & 1 & 2 & 3 \\
	0 & 1 & 2 & 3 & 4 \\
	1 & 2 & 3 & 4 & 0 \\
\end{matrix}
$\hspace{30pt}$
\begin{matrix}
	3 & 0 & 2 & 4 & 1 \\
	4 & 1 & 3 & 0 & 2 \\
	0 & 2 & 4 & 1 & 3 \\
	1 & 3 & 0 & 2 & 4 \\
	2 & 4 & 1 & 3 & 0 \\
\end{matrix}
$\hspace{30pt}$
\begin{matrix}
	4 & 2 & 0 & 3 & 1 \\
	0 & 3 & 1 & 4 & 2 \\
	1 & 4 & 2 & 0 & 3 \\
	2 & 0 & 3 & 1 & 4 \\
	3 & 1 & 4 & 2 & 0 \\
\end{matrix}
$\hspace{30pt}$
\begin{matrix}
	0 & 4 & 3 & 2 & 1 \\
	1 & 0 & 4 & 3 & 2 \\
	2 & 1 & 0 & 4 & 3 \\
	3 & 2 & 1 & 0 & 4 \\
	4 & 3 & 2 & 1 & 0 \\
\end{matrix}
$
\\
hledal jsem pomoci vzorce viz prvni uloha, JS kod nize
\begin{verbatim}
A=(k,n)=>{
    const mat=[]
    for(let i = 0;i<n;i++){
        mat[i]=[]
        for(let j = 0;j<n;j++)
            mat[i][j] = (i+1+k*j+k)%n
    }
    return mat
}
\end{verbatim}


\section*{Treti ukol}
$15*7/3 = 35$\\
protoze kazde dite musi mit prave 7 dni sluzbu\\
avsak kazdou sluzbu bychom takto zapocitali 3krat




\section*{Ctvrty ukol}
Pouzijeme Vandermondovu konvoluci\\
$
\binom{m+n}{r}=\sum^r_{k=0}\binom{m}{k}\binom{n}{r-k}
$\\
$m=n=r$\\
$
\binom{2n}{n}=\sum^r_{k=0}\binom{n}{k}\binom{n}{n-k}
$\\





\section*{Paty ukol}
prevedeme na konecnou projektivni rovinu,\\
mejme body jako ucitele,\\
primky jako studenty\\
a nech kazda primka je dlouha $k$\\
dostaneme KPR radu $h$.









\end{document}