\documentclass[a4paper]{article}
\usepackage[utf8]{inputenc}
\usepackage{amsmath}
\usepackage{amsfonts}
\usepackage{amssymb}
\usepackage{hyperref}
\usepackage{fancyhdr}
\usepackage[czech]{babel} % recommended if you write in Czech
\usepackage[
    left   = 1.0 in,
    right  = 1.0 in,
    top    = 1.5 in,
    bottom = 1.5 in,
]{geometry}
\setlength{\parindent}{0in}

%%% -------------------------------------------------
%%% zpracovano Davidem Napravnikem pro potreby MFF UK
%%% -------------------------------------------------

\begin{document}

\pagestyle{fancy}
\rhead{Maxim Dokonaly}

\section*{1. HW}
\subsection*{1. priklad}

\textbf{tucny text}

\textit{kurziva}

\begin{itemize}
    \item necislovany seznam
\end{itemize}

\begin{enumerate}
    \item cislovany seznam
\end{enumerate}

% poznaka co se nevygeneruje zacina procentem

jeden radek
lze prerusti

prazdnym radkem \\
nebo dvoj-lomitky

matematicka rovnice v radku $a=3$ neboli inline

Matematicka rovnice samostatne:
$$a=3$$

rychla url: \url{www.google.com}

url s alternativnim textem (klikaci odkaz zmizi po vytisteni): \href{https://google.com}{\nolinkurl{google.com}}

poznamka \footnote{pod carou}

tabulky lze snadno vygenerovat zde: \url{https://www.tablesgenerator.com/}

\begin{table}[h]
    \begin{tabular}{|l|c|r|} % lcr znaci zarovani pro kazdy sloupec a svislitka ohraniceni
    l jako vlevo & c jako uprostred & r jako vpravo \\
    1 & 2 & 3 \\ \hline % hline pro horizontalni cary
    4 & 5 & 6
    \end{tabular}
\end{table}

pouze v matematickem mode muzeme pouzit dalsi symboly:
$$\alpha+\beta-\pi=\forall<\emptyset$$
$$a^2 \leftarrow b_2 \Leftarrow c^1_2 \neq d^{-1}_{-2}$$

vetsi funkce, ktere vypadaji jinak inline: $\sum^{\infty}_{n=1} \frac{1}{n} = \infty$ 
$$\sum^{\infty}_{n=1} \frac{1}{n} = \infty$$

a na konci dukazu pouzijeme \hfill$\square$

dost symbolu tu ma vyznam, tak je nezapomente \uv{escapovat}: \{ \} \# \% \$
a jedna vyjimka na konec: $\backslash$


\end{document}